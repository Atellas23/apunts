\documentclass[10pt]{article}
\usepackage[utf8]{inputenc}
\usepackage{amsfonts}
\usepackage{amsmath,amssymb}
\usepackage{amsthm}
\usepackage{float}
\usepackage[margin=1in]{geometry}
\usepackage{color}
\usepackage{breqn}
\newcommand{\af}{\mathbb{A}}
\newcommand{\euc}[1]{\mathbb{E}_{\mathbb{R}}^{#1}}
\newcommand{\norm}[1]{||#1||}
\newcommand{\scal}[2]{\left<#1,#2\right>}
\newcommand{\obs}{\underline{\textbf{Observació}}: }
\DeclareMathOperator{\idn}{Id}
\author{Àlex Batlle Casellas}
\title{Randomly solving \textsc{2-SAT}.}
\date{}

\begin{document}

\maketitle

\paragraph{a)} Let $X_n$ be the number of satisfied clauses after $n$ iterations of the loop in (2) (we will refer to it as time $n$). Given the execution up top time $n-1$, is it always true that $\mathbb{E}\left[X_n\right]\geq X_{n-1}$?\\
\textbf{Solution}\\
The claim is not true. Let us prove it by a counterexample: let $\phi$ be the following CNF formula: \[\phi=(x_1\vee x_2)\wedge(x_1\vee x_3)\wedge(\bar{x_1}\vee\bar{x_4})\wedge(\bar{x_1}\vee x_3)\wedge(\bar{x_4}\vee\bar{x_3}),\] and $\textbf{x}=(0,1,0,1)$ the assignment at time $n-1$. Then, it follows that $X_{n-1}=4$, and the expectancy of the variable $X_n$ can be calculated as follows; whether $x_1$ swaps its value, and then $X_n=3$, or $x_3$ swaps its value and $X_n=4$. This gives an expectancy
\[\mathbb{E}\left[X_n\right]=\dfrac{1}{2}3+\dfrac{1}{2}4=\dfrac{7}{2}<4=X_{n-1}.\]

\paragraph{b)} Since $\phi$ is satisfiable, let $\textbf{x}^*$ be an arbitrary satisfying assignment. Let $Y_n$ be the number of variables in $\textbf{x}$ whose value coincides with the one in $\textbf{x}^*$ at time $n$. Given the execution up to time $n-1$, is it always true that $\mathbb{E}\left[Y_n\right]\geq X_{n-1}$?\\
\textbf{Solution}\\

\paragraph{c)} Argue that if $Y_n=k$, then \textsc{Rand2SAT} terminates at time $n$. Is the converse true?\\
\textbf{Solution}\\
Since $Y_n=k$, it means that all variables in $\textbf{x}$ are equal to $\textbf{x}^*$. As $\textbf{x}^*$ is an arbitrary satisfying assignment, it means that $\phi(\textbf{x}^*)=1$, and as $\phi\in\textsc{CNF-2-SAT}$, it follows that is has no violated clauses. Then, as described by the algorithm, the loop in (2) stops when no clauses are violated. Hence, when $Y_n=k$, the algorithm halts at this moment (\textit{time $n$}).

\paragraph{d)} Is $Y_n$ a Markov chain?\\
\textbf{Solution}\\

\paragraph{e)} Design a Markov chain $Z_n$ such that $Y_n\geq Z_n$.\\
\textbf{Solution}\\

\paragraph{f)} Use $Z_n$ to prove that the expected running time of \textsc{Rand2SAT} is at most $k^2$.\\
\textbf{Solution}\\

\paragraph{*g)} We modify \textsc{Rand2SAT} to stop in bounded time as follows. Let $l\in\mathbb{Z}$. If after $2lk^2$ iterations of the loop in (2) we have not halted, we break the loop and return the current assignment $\textbf{x}$. Prove that the output of the modified \textsc{Rand2SAT} is a satisfying assignment with probability at least $1-2^{-l}$.\\
\textbf{Solution}\\
\end{document}