\documentclass[11pt]{article}
\usepackage[catalan]{babel}
\usepackage[utf8]{inputenc}
\usepackage{amsfonts}
\usepackage{amsmath,amssymb}
\usepackage[margin=1.25in]{geometry}
\usepackage{color}
\usepackage{breqn}
\newcommand{\scal}[2]{\left<#1,#2\right>}
\newcommand{\norm}[1]{||#1||_2}
\newcommand{\norminf}[1]{||#1||_{\infty}}
\title{Segona Pràctica d'Àlgebra Lineal Numèrica}
\author{Àlex Batlle Casellas}

\begin{document}
\maketitle
\section{Demostració.}
\paragraph{Objectiu.} Volem demostrar que
\[
\sqrt{\sum_{i=1}^m(p_q(x_i)-y_i)^2}\leq\sqrt{\sum_{i=1}^m(r_q(x_i)-y_i)^2}\quad\forall r_q(x)\in\mathbb{R}_q[x]
\]
\[\iff p_q(x)\text{ és solució de les equacions normals.}\]
Observem que demostrar això es equivalent a demostrar que el vector $a^*\in\mathbb{R}^{q+1}$ minimitza $||Aa-y||_2$ per tota $a\in\mathbb{R}^{q+1}$ si i només si $A^t(Aa^*-y)=0$, amb $A\in\mathcal{M}_{m\times q+1}(\mathbb{R})$ les últimes $q+1$ columnes de la matriu de Vandermonde associada al vector $x$ donat. Aquesta demostració la vam fer a classe i la reproduïm aquí:
\begin{itemize}
    \item[$\impliedby)$] Suposem que $a^*$ compleix $A^t(Aa^*-y)=0$. Aleshores, prenem $a\in\mathbb{R}^{q+1}$, i cal veure que $||Aa-y||_2\geq||Aa^*-y||_2$. Veurem un resultat equivalent amb els quadrats respectius de cada costat, i tenint en compte que $\norm{u-v}=|-1|\norm{v-u}=\norm{v-u}$. Escrivim $y-Aa=y-Aa^*+A(a^*-a)$. Aleshores,
    \[
    ||y-Aa||_2^2=\scal{y-Aa}{y-Aa}=\scal{y-Aa^*+A(a^*-a)}{y-Aa^*+A(a^*-a)}=\]
    \[\norm{y-Aa^*}^2+\norm{A(a^*-a)}^2+2\scal{y-Aa^*}{A(a^*-a)}=\norm{y-Aa^*}^2+\norm{A(a^*-a)}^2+\]
    \[2\scal{A^t(y-Aa^*)}{a^*-a}\geq\norm{y-Aa^*}^2.
    \]
    \item[$\implies)$] Suposem que $a^*$ minimitza la norma que volem i que $A^t(y-Aa^*)=z\neq0$, i arribem a un absurd. $\forall\epsilon>0$ prenem $a=a^*+\epsilon z$. Aleshores,
    \[
    \norm{y-Aa}^2=\norm{y-A(a^*+\epsilon z)}^2=\scal{y-Aa^*-\epsilon Az}{y-Aa^*-\epsilon Az}=\norm{y-Aa^*}^2
    \]
    \[
    +\epsilon^2\norm{Az}-2\epsilon\scal{y-Aa^*}{Az}=\norm{y-Aa^*}^2+\epsilon^2\norm{Az}^2-2\epsilon\norm{z}^2=\norm{y-Aa^*}^2-\]
    \[\epsilon(2\norm{z}^2-\epsilon\norm{Az}^2)\implies\norm{y-Aa}^2<\norm{y-Aa^*},\text{ on tenim una contradicció.}
    \]
    Per tant, agafant $a^*$ la solució de les equacions normals, queda demostrada la proposició. $\square$
\end{itemize}
\section{Justificació del mètode emprat.}
Hem utilitzat el mètode de descomposició $A=QR$ per trobar l'ajustament per mínims quadrats. Aquest està justificat doncs, si plantegem altre cop les equacions en forma matricial, tenim
\begin{equation}
    \begin{pmatrix}
    x_1^q & x_1^{q-1} & \cdots & x_1 & 1\\
    x_2^q & x_2^{q-1} & \cdots & x_2 & 1\\
    \vdots & \vdots & \ddots & \vdots & \vdots\\
    x_m^q & x_m^{q-1} & \cdots & x_m & 1\\
    \end{pmatrix}
    \begin{pmatrix}
    a_q\\
    a_{q-1}\\
    \vdots\\
    a_1\\
    a_0
    \end{pmatrix}
    =\begin{pmatrix}
    y_1\\
    y_2\\
    \vdots\\
    y_m
    \end{pmatrix}.
\end{equation}
Les equacions normals d'aquest sistema són de la forma (dient $A$ a la matriu del sistema, $a$ al vector de coeficients i $y$ al vector de valors de la funció) $A^tAa=A^ty$. Si descomposem $A=QR$ i tornem a escriure les equacions normals, queda (sabent que $Q$ és ortogonal)
\[
(QR)^tQRa=(QR)^ty\iff R^tQ^tQRa=R^tQ^ty\iff Ra=Q^ty,
\]
i per tant, resoldre el sistema triangular $Ra=Q^ty$ és equivalent a resoldre $A^tAa=A^ty$.
\section{Especificació de la funció \texttt{polminquad}.}
\paragraph{Objectius.} La funció \texttt{polminquad} té com a objectiu calcular l'ajustament polinòmic pel mètode de mínims quadrats d'una taula de dades $(x_i,y_i)_{i=1,\ldots,m}$ donada. La funció utilitza la descomposició $QR$ per resoldre el sistema d'equacions normals, i per comprovar fins a quin nivell la descomposició és efectiva, treu per pantalla la norma del residu $\norminf{Q^tQ-Id_{q+1}}$ (variable \texttt{residu_Q_ortogonal}). Si es demana, la funció dibuixa el polinomi trobat.
\paragraph{Paràmetres d'entrada.}
\begin{itemize}
    \item[\texttt{x}:] vector d'abcisses de la taula de dades donada.
    \item[\texttt{y}:] vector d'ordenades de la taula de dades donada.
    \item[\texttt{grau}:] grau del polinomi que es vol trobar pel mètode de mínims quadrats.
    \item[\texttt{plt}:] paràmetre d'entrada opcional. Si apareix, indica el nombre de punts equiespaiats amb els que es dibuixarà la gràfica del polinomi.
\end{itemize}
\paragraph{Paràmetres de sortida.}
\begin{itemize}
    \item[\texttt{coefs}:] vector de coeficients del polinomi trobat.
    \item[\texttt{norm2Res}:] norma-2 del residu que dona el polinomi trobat; si $a^*$ identifica el vector \texttt{coefs}, aquest paràmetre de sortida val $\norm{Aa^*-y}$.
\end{itemize}
\end{document}
