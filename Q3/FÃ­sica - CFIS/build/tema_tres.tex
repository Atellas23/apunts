\chapter{Dinàmica de la partícula}

\section{Principis de la mecànica clàssica}
%La dinàmica estudia el moviment de les partícules degut a interaccions amb els cossos que l'envolten. Per calcular el moviment degut a aquestes interaccions, hem d'introduïr el concepte de força. Coneguda la força, el moviment es regeix per les 3 lleis de Newton:
\begin{itemize}
\item[I.] Tot cos es manté en un estat de repòs o de moviment rectilini uniforme excepte si se l'obliga a variar aquest estat mitjançant forces que actuen sobre ell.
\item[II.] La variació del moviment és proporcional a la força que actua sobre el cos i s'efectua en la direcció de la recta en la que actua la força.
\item[III.] A tota acció s'oposa sempre una reacció igual.
\end{itemize}
\subsection{La primera llei de Newton: la llei de la inèrcia}
%Ja fou formulada per Galileu. Defineix de manera operacional el que és una força.
%Quan diem que una partícula, sense l'acció d'una força, es troba en repòs o en moviment rectilini, hem de dir respecte qui es troba en aquest estat. Ha de ser respecte una altra partícula lliure. Tal \textit{observador} és un observador \textit{inercial}, i el sistema de referència associat és un \textit{sistema de referència inercial}.
%La Terra, per exemple, no és un referencial inercial, però per la majoria dels problemes d'interès la podem considerar com a tal.
\subsubsection{Transformacions de Galileu}
Tots els referencials inercials són equivalents. Siguin dos referencials inercials $S$ i $S'$ que es mouen un respecte l'altre amb velocitat uniforme, de manera que el vector entre els seus origens $O$ i $O'$ és $\vec{OO'}=\vec v_0 t$. La recta de posició del punt $P$ està relacionada per $\vec r_P=\vec{r_{P}'}+\vec v_0  t$, on suposarem que $t=t'$, de manera que la mesura del temps és la mateixa a $S$ i a $S'$. Això és la transformació de Galileu. Derivant un cop,
\[\vec v'=\dfrac{d\vec r'}{dt'}=\dfrac{d\vec r'}{dt},\quad \vec v=\dfrac{d\vec r}{dt}=\dfrac{d\vec r'}{dt}+\vec v_0=\vec v' + \vec v_0.\]
Per l'acceleració,
\[\vec a'=\dfrac{d\vec v'}{dt'},\quad \vec a=\dfrac{d\vec v}{dt}\implies\vec a=\vec a'.\]
L'acceleració queda invariant. D'aquí arribem al principi de relativitat de Galileu: les lleis bàsiques de la física són idèntiques en totes els referencials que es mouen amb velocitat uniforme els uns respecte dels altres.

\subsubsection{Transformacions de Lorentz}
No imposarem que $t=t'$, però sí que la velocitat de la llum és la mateixa mesurada en qualsevol referencial inercial. Un dels sistemes es mou amb velocitat $v$ respecte de l'altre.
%Aquí passen coses màgiques que estan als apunts d'en Blas
Si fem 
\[
\gamma=\dfrac{1}{\sqrt{1-\dfrac{v^2}{c^2}}},
\]
aleshores, les transformacions de Lorentz són
\[\begin{cases}x'=\gamma(x-vt)\\ y'=y\\ z'=z\\ t'=
\gamma(t-\dfrac{xv}{c^2})
\end{cases}\]

\subsection{La segona llei de Newton}
Actualment, l'escrivim com $\vec F=m\vec a$. Si tenim que la mateixa força provoca acceleracions diferents en cossos diferents, podem definir el quocient de les seves masses (la seva inèrcia, o oposició al moviment) com
\[\dfrac{m_1}{m_2}=\dfrac{a_2}{a_1}.\]
A partir d'una massa patró podem definir la resta.\\
Tantmateix, l'enunciat original de Newton no parla ni de massa ni d'acceleració, sinó de variació de moviment. Podem definir la quantitat de moviment com $\vec p=m\vec v$, de manera que podem escriure la segona llei de Newton com
\[\vec F=\dfrac{d\vec p}{dt}=m\dfrac{d\vec v}{dt}=m\vec a.\]
Per una partícula puntual passa això, però no per un sistema de partícules. En mecànica relativista, coses.
%inserir coses

\subsection{La tercera llei de Newton}
Si un cos $A$ exerceix una força sobre un cos $B$, $\vec F_{AB}$, aleshores $B$ exerceix una força $\vec F_{BA}$ sobre $A$, tal que $\vec F_{BA}=-\vec F_{AB}$.
\subsubsection{Conservació de la quantitat de moviment}
Siguin dues partícules aïllades que exerceixen forces entre elles. Per la segona llei de Newton,
\[\vec F_1=\dfrac{d\vec p_1}{dt},\quad\vec F_2=\dfrac{d\vec p_2}{dt}.\]
Però, segons la tercera llei,
\[\vec F_1+\vec F_2=0\implies\dfrac{d\vec p_1}{dt}+\dfrac{d\vec p_2}{dt}=0\implies\vec p_1+\vec p_2=\text{const}.\]

\subsection{Forces de fricció}
Dos cossos en contacte presenten una força entre ells que s'oposa a que llisquin. Aquesta força és proporcional a la força normal (la força de reacció del terra al pes del cos), $F_R\propto N$, $F_R=\mu N$. La força de fregament s'oposa al moviment, i per tant és $\vec F_R=-\mu N\vec v$. Existeixen dos coeficients de fregament,
\begin{itemize}
	\item $\mu_s\longrightarrow$ coeficient de fregament estàtic
	\item $\mu_d\longrightarrow$ coeficient de fregament dinàmic
\end{itemize}
\subsubsection{Forces de fregament en fluids}
Les forces de fregament en fluids normalment són proporcionals a la velocitat, $\vec F_R=-k\eta\vec v$. $\eta$ és el \textit{coeficient de viscositat}, i $k$ és un factor geomètric; per una esfera, $k=6\pi r$, $\vec F_R=-6\pi r\eta\vec v$ (l'anomenada Força d'Stokes).

\subsection{Sistemes de massa variable}
\[\vec F=\dfrac{d\vec p}{dt}=\dfrac{d}{dt}(m\vec v)=\dfrac{dm}{dt}\vec v+\dfrac{d\vec v}{dt}m.\]
Si les forces externes són zero, la quantitat de moviment es conserva.

\section{Moviment curvilini}
Si la força no és proporcional a la velocitat, aleshores el moviment descrit serà curvilini. Com que tenim $\vec F=m\vec a$, aleshores podem definir les components tangencial i normal de la força, de manera que:
\[
\begin{cases}
F_t=ma_t=m\dfrac{dv}{dt}\\
F_n=ma_n=\dfrac{mv^2}{\rho}
\end{cases}.
\]
En el cas de moviment circular, $\rho=R,\ v=\omega R$, i les forces són
\[
\begin{cases}
F_t=mR\dfrac{d\omega}{dt}\\
F_n=m\dfrac{m\omega^2R^2}{R}=m\omega^2R
\end{cases}.
\]
Si a més, $\omega$ és uniforme,
\[
\begin{cases}
F_t=0\\
F_n=m\omega^2R
\end{cases}\implies\vec a=\vec\omega\times\vec v,\ \vec F=m\vec a=\vec\omega\times m\vec v=\vec\omega\times\vec p.
\]