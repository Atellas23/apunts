\chapter{Cinemàtica de la partícula}

\section{Moviment}
L'objectiu d'aquesta secció és descriure el moviment d'una partícula mitjançant expressions matemàtiques. Per tal de fer això, necessitarem un espai amb
\begin{itemize}
	\item un origen;
	\item una base vectorial.
\end{itemize}
\begin{defi}[Moviment]
    El moviment d'una partícula és un vector $\vec{r}$ que uneix l'origen de coordenades amb la posició de la partícula, i que pot dependre de diversos paràmetres. Si depèn del paràmetre $t$, en coordenades cartesianes el podem escriure com
    \[
    \vec{r}(t)=x(t)\hat{\mathbf{i}}+y(t)\hat{\mathbf{j}}+z(t)\hat{\mathbf{k}}.
    \]
\end{defi}
\begin{defi}[Velocitat]
	La velocitat d'una partícula amb moviment $\vec{r}$ és la seva derivada respecte el temps,
	\[
	\vec{v}=\lim_{\Delta t\rightarrow 0}\dfrac{\Delta\vec{r}}{\Delta t}=\dfrac{d\vec{r}}{dt}=\dot{\vec{r}}.
	\]
\end{defi}
La velocitat sempre és tangent a la trajectòria, i per tant, si $\hat{\mathbf{t}}$ és el vector unitari en la direcció tangent a $\vec{r}$, tenim $\vec{v}=v\hat{\mathbf{t}}$. En coordenades cartesianes,
\[
\dfrac{d\vec{r}}{dt}=\dfrac{dx}{dt}\hat{\mathbf{i}}+\dfrac{dy}{dt}\hat{\mathbf{j}}+\dfrac{dz}{dt}\hat{\mathbf{k}}.
\]
\begin{defi}[Acceleració]
	L'acceleració d'una partícula amb velocitat $\vec{v}$ és la seva derivada respecte el temps,
	\[
	\vec{a}=\dot{\vec{v}}=\lim_{\Delta t\rightarrow 0}\dfrac{\Delta\vec{v}}{\Delta t}=\dfrac{d^2\vec{r}}{dt^2}=\ddot{\vec{r}}.
	\]
\end{defi}
\newpage

\section{Canvis de coordenades}
\subsection{Coordenades polars}
\[
\begin{cases}
\hat{\mathbf{r}}=\hat{\mathbf{i}}\cos\theta+\hat{\mathbf{j}}\sin\theta\\
\hat{\mathbf{\theta}}=-\hat{\mathbf{i}}\sin\theta+\hat{\mathbf{j}}\cos\theta
\end{cases}\iff
\begin{pmatrix}
\cos\theta & \sin\theta\\
-\sin\theta & \cos\theta
\end{pmatrix}\begin{pmatrix}\hat{\mathbf{i}}\\ \hat{\mathbf{j}}\end{pmatrix}=
\begin{pmatrix}\hat{\mathbf{r}}\\ \hat{\mathbf{\theta}}\end{pmatrix}
\iff
\begin{pmatrix}
\cos\theta & -\sin\theta\\
\sin\theta & \cos\theta
\end{pmatrix}\begin{pmatrix}\hat{\mathbf{r}}\\ \hat{\mathbf{\theta}}\end{pmatrix}=
\begin{pmatrix}\hat{\mathbf{i}}\\ \hat{\mathbf{j}}\end{pmatrix}
\]
\textbf{Canvis en la velocitat i l'acceleració}
\begin{itemize}
	\item Velocitat:
		\[
		\begin{cases}
		v_r=\dot r \\
		v_\theta=r\dot\theta
		\end{cases}
		\]
	\item Acceleració:
		\[
		\begin{cases}
		a_r=\ddot r - r\dot\theta^2 \\
		a_\theta=r\ddot\theta + 2\dot r\dot\theta
		\end{cases}
		\]
\end{itemize}

\subsection{Coordenades intrínseques}
\[\vec{r}(t),\ \vec{v}=\dfrac{d\vec r}{dt}=v\hat t,\ \vec{a}=\dfrac{d\vec v}{dt}=\dfrac{d}{dt}(v\hat t)=\dfrac{dv}{dt}\hat t + v\dfrac{d\hat t}{dt}.\]
Tenim que $\hat t\cdot\hat t=1$, i aleshores 
\[
\dfrac{d}{dt}(\hat t\cdot\hat t)=0\implies2\hat t\dfrac{d\hat t}{dt}=0\implies\dfrac{d\hat t}{dt}\perp\hat t.
\]
Si considerem un diferencial d'arc de la trajectòria, tenim que localment $ds=\rho d\theta$. També, el mòdul del vector tangent $\hat t$ és localment $|d\hat t|=d\theta$. Sabent que $\dfrac{d\hat t}{dt}=a\hat n$, aleshores
\[\left|\dfrac{d\hat t}{dt}\right|=\left|\dfrac{d\hat t}{ds}\right|\dfrac{ds}{dt}=v\dfrac{d\theta}{ds}=\dfrac{v}{\rho}\implies
\dfrac{d\hat t}{dt}=\dfrac{v}{\rho}\hat n.\]
\textbf{Canvis en la velocitat i l'acceleració}
\[\vec a=\dfrac{dv}{dt}\hat t + \dfrac{v^2}{\rho}\hat n=a_t\hat t + a_n\hat n,\ a_t=\dfrac{dv}{dt},\ a_n=\dfrac{v^2}{\rho}.\]
Si sabem $a$ i $v$,
\[\begin{cases}v\cdot a=v\cdot(\vec a_t + \vec a_n)=va_t \\ v\times a=va_n(\hat t\times\hat n)\end{cases}\implies\begin{cases}a_t=\dfrac{\vec v\cdot\vec a}{v}\\ a_n=\dfrac{|\vec v\times\vec a|}{v}\\ \rho=\dfrac{v^3}{|\vec v\times\vec a|}\end{cases}\]

\section{Tipus de moviments}
\subsection{Uniforme}
$\vec v=\vec v_0$ és constant, $\vec r(t)=\int\vec v dt=\vec v_0 t+\vec r_0$ és una recta. 
\subsection{Uniformement accelerat}
$\vec a=\vec a_0$ constant, per tant
\[\vec v=\int\vec a dt=\vec a_0t+\vec v_0\implies\vec r(t)=\int\vec v dt=\int(\vec a_0t+\vec v_0)dt=\vec a_0\dfrac{t^2}{2}+\vec v_0 t+\vec r_0.\]
$\vec v$ es troba sempre en el pla definit per $\vec v_0$ i $\vec a_0$, i $\vec r - \vec r_0$ també es troba en aquest mateix pla$\implies$ el moviment és sempre en el mateix pla.
\subsection{Circular}
La velocitat és tangent a la circumferència que descriu el moviment, $\vec v\perp\vec r$, i $|\vec r|=R$ constant. És habitual representar la velocitat angular amb un vector $\omega$, de direcció l'eix de rotació i sentit governat per la regla de la mà dreta. En aquest cas, $\vec v=\vec\omega\times\vec r$. La velocitat i l'acceleració són, respectivament,
\[
\text{Velocitat:}
\begin{cases}
v_r=\dot r=0\\
v_\theta=r\dot\theta=\omega R
\end{cases}
;\quad
\text{Acceleració:}
\begin{cases}
a_r=\ddot r - r\dot\theta^2=-r\omega^2\\
a_\theta=r\ddot\theta + 2\dot r\dot\theta=r\dot\omega
\end{cases}.
\]
Si $\omega$ és constant, tenim $a_\theta=0$, i aleshores,\[\vec a=\dfrac{d\vec v}{dt}=\dfrac{d}{dt}(\vec\omega\times\vec r)=\vec\omega\times\dfrac{d\vec r}{dt}=\vec\omega\times\vec v=\vec\omega\times(\vec\omega\times\vec r).\]