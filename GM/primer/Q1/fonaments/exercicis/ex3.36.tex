\documentclass[11pt]{article}
\usepackage[utf8]{inputenc}
\usepackage{amsfonts}
\usepackage{amsmath}
\usepackage{amsthm}
\usepackage{amssymb}
\usepackage{float}
\usepackage[margin=3cm]{geometry}
\usepackage{color}
\usepackage{enumitem}
\usepackage{calrsfs}
\title{Exercicis: Fonaments de les Matemàtiques}
\author{Àlex Batlle Casellas}
\renewcommand*\contentsname{Índex}
\newlist{legal}{enumerate}{10}
\setlist[legal]{label*=\arabic*.}
\newcommand{\pendent}{\textbf{\textcolor{red}{PENDENT D'ACABAR.}}\\}
\DeclareMathAlphabet{\pazocal}{OMS}{zplm}{m}{n}

\begin{document}

\begin{legal}
	\item[3.36.] Considereu les permutacions de $\mathfrak{S}_5$ següents:
	$$
	\sigma=
	\begin{pmatrix}
		1 & 2 & 3 & 4 & 5\\
		3 & 5 & 4 & 1 & 2
	\end{pmatrix},
	\qquad
	\tau=
	\begin{pmatrix}
		1 & 2 & 3 & 4 & 5\\
		5 & 3 & 4 & 2 & 1
	\end{pmatrix},
	\qquad
	\rho=
	\begin{pmatrix}
		1 & 2 & 3 & 4 & 5\\
		3 & 2 & 4 & 5 & 1
	\end{pmatrix}
	$$
	\begin{legal}
		\item[1) ]\textbf{Descomponeu les tres permutacions en producte de cicles.}
		$$\sigma=
		\begin{pmatrix}
			1 & 2 & 3 & 4 & 5\\
			3 & 5 & 4 & 1 & 2
		\end{pmatrix}
		=(1,3,4)(2,5).
		$$
		$$\tau=
		\begin{pmatrix}
			1 & 2 & 3 & 4 & 5\\
			5 & 3 & 4 & 2 & 1
		\end{pmatrix}
		=(1,5)(2,3,4).
		$$
		$$\rho=
		\begin{pmatrix}
			1 & 2 & 3 & 4 & 5\\
			3 & 2 & 4 & 5 & 1
		\end{pmatrix}
		=(1,3,4,5).
		$$
		\item[2) ]\textbf{Calculeu $\sigma\tau\rho$ i $\sigma\rho^2$.}
		$$
		\sigma\tau\rho=\begin{pmatrix}
		1 & 2 & 3 & 4 & 5\\
		1 & 4 & 5 & 3 & 2
		\end{pmatrix}.
		$$
		$$
		\sigma\rho^2=\begin{pmatrix}
		1 & 2 & 3 & 4 & 5\\
		1 & 5 & 2 & 3 & 4
		\end{pmatrix}.
		$$
		\item[3) ]\textbf{Trobeu la signatura de $\tau$ i de $\rho^{-1}$.}\\
		Es pot descomposar $\tau$ en tres transposicions:
		$$\tau=(1,5)(2,3)(3,4),$$
		i per tant $\mathcal{E}(\tau)=(-1)^3=-1.$\\
		Per descompondre $\rho^{-1}$, primer hem de trobar aquesta permutació. Seguint el recorregut de $\rho$, només cal intercanviar les files i reordenar per obtenir la inversa. Ho fem:
		$$\rho^{-1}=\begin{pmatrix}
		1 & 2 & 3 & 4 & 5\\
		5 & 2 & 1 & 3 & 4
		\end{pmatrix}.
		$$
		Per tant, si ara descomposem $\rho^{-1}$ en producte de transposicions en podem trobar la signatura:
		$$
		\rho^{-1}=\begin{pmatrix}
		1 & 2 & 3 & 4 & 5\\
		5 & 2 & 1 & 3 & 4
		\end{pmatrix}
		=(1,5,4,3).
		$$
		Per tant, $\mathcal{E}(\rho^{-1})=(-1)^1=-1$.
	\end{legal}
\end{legal}
\end{document}
