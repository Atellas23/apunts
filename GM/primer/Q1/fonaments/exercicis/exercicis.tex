\documentclass[11pt]{article}
\usepackage[utf8]{inputenc}
\usepackage{amsfonts}
\usepackage{amsmath}
\usepackage{amsthm}
\usepackage{amssymb}
\usepackage{float}
\usepackage[margin=3cm]{geometry}
\usepackage{color}
\usepackage{enumitem}
\title{Exercicis: Fonaments de les Matemàtiques}
\author{Àlex Batlle Casellas}
\renewcommand*\contentsname{Índex}
\newlist{legal}{enumerate}{10}
\setlist[legal]{label*=\arabic*.}
\newcommand{\pendent}{\textbf{\textcolor{red}{PENDENT D'ACABAR.}}\\}

\begin{document}

\begin{titlepage}
	\centering
	{\scshape\LARGE Facultat de Matemàtiques i Estadística \par}
	\vspace{1cm}
	{\scshape\Large Universitat Politècnica de Catalunya - BarcelonaTech\par}
	\vspace{1.5cm}
	{\huge\bfseries Exercicis resolts de Fonaments de les Matemàtiques (Primer curs del Grau de Matemàtiques)
	\par}
	\vspace{2cm}
	{\Large\itshape Àlex Batlle Casellas\par}

	\vfill

% Bottom of the page
	{\large \today\par}
\end{titlepage}

%\section*{Resumen}

\vfill
\newpage

\tableofcontents
\newpage
\section{Formalisme matemàtic: enunciats i demostracions.}

\newpage

\section{Conjunts i aplicacions.}
\begin{legal}[start=21]
	\item \textbf{Siguin $A_1,A_2,B_1,B_2\neq\emptyset$. Demostreu:}
	\begin{legal}[start=3]
		\item \textbf{$(A_1\cup A_2)\times (B_1\cup B_2)=(A_1\times B_1)\cup (A_1\times B_2)\cup (A_2\times B_1)\cup (A_2\times B_2)$.}
		$$\textrm{Sigui }y\in (A_1\cup A_2)\times (B_1\cup B_2).\textrm{ Aleshores, }\exists y_1\in A_1\cup A_2, \ y_2\in B_1\cup B_2: \ y=(y_1,y_2).$$
		$$\iff (y_1\in A_1 \vee y_1\in A_2)\wedge (y_2\in B_1 \vee y_2\in B_2)\iff (y_1\in A_1\wedge y_2\in B_1)$$
		$$\vee (y_1\in A_2 \wedge y_2\in B_1)\wedge (y_1\in A_1\wedge y_2\in B_2)\vee (y_1\in A_2\wedge y_2\in B_2)$$
		$$\iff y\in (A_1\times B_1)\cup (A_2\times B_2)\cup (A_1\times B_2)\cup (A_2\times B_1).\square$$
	\end{legal}
	\item[30.] \textbf{Considerem una aplicació $f: \ A\mapsto B$ i subconjunts $A',A''\subseteq A$ i $B',B''\subseteq B$.}
	\begin{legal}
		\item[30.1.] \textbf{Demostreu que si $A'\subseteq A''$, aleshores $f(A')\subseteq f(A'')$.}
		$$\textrm{Sigui }A'\subseteq A''.\textrm{ Aleshores }f(A')=\{y\in B:(\exists x\in A':\ f(x)=y)\}$$
		$$\subseteq \{y\in B:(\exists x\in A'':f(x)=y)\}=f(A'')\implies f(A')\subseteq f(A'').\square$$
		%Si $f$ és injectiva, volem veure que $f(A'')\subseteq f(A')$ (ja que la primera inclusió per la igualtat ja l'hem demostrada a l'apartat anterior).
		%$$\textrm{Sigui }A'\subseteq A''.\textrm{ Aleshores }f(A'')\subseteq f(A') \iff A''\subseteq A'\textrm{ (resultat anterior) }\iff$$
		%$$A''=A'\textrm{ (perquè sabem }A'\subseteq A''\textrm{) }\iff\textrm{ (sabent que }f(A')\subseteq f(A'')\textrm{) }f\textrm{ és injectiva.}\square$$
		\item[30.3.] \textbf{Demostreu que si $B'\subseteq B''$, aleshores $f^{-1}(B')\subseteq f^{-1}(B'')$.}
		$$\textrm{Sigui }B'\subseteq B''.\textrm{ Aleshores, }f^{-1}(B')=\{x\in A:(\exists y\in B':f^{-1}(\{y\})=\{x\})\}\subseteq$$
		$$\{x\in A:(\exists y\in B'':f^{-1}(\{y\})=\{x\})\}=f^{-1}(B'')\implies f^{-1}(B')\subseteq f^{-1}(B'').\square$$		
		\item[30.2.] \textbf{Demostreu que $f(A')\subseteq f(A'')$ implica que $A'\subseteq A''$, per a tot $A',A''\subseteq A$, si, i només si, $f$ és
injectiva.}\\
		Si fem el conjunt antiimatge dels dos costats de la hipòtesi ($f(A')\subseteq f(A'')$):
		$$f^{-1}(f(A'))\subseteq f^{-1}(f(A''))\implies (f^{-1}\circ f)(A')\subseteq (f^{-1}\circ f)(A'')\implies$$
		$$Id_A(A')\subseteq Id_A(A'')\implies A'\subseteq A''.$$
		Això només passarà quan $f$ és injectiva, doncs en tal cas $A'$ i $A''$ no podrien ser disjunts.$\square$\\Si $f$ no fos injectiva, en canvi, $A'$ i $A''$ podrien ser disjunts però donar el mateix conjunt imatge sense inconvenient.
		\item[30.4.] \textbf{Demostreu que $f^{-1}(B')\subseteq f^{-1}(B'')$ implica que $B'\subseteq B''$, per a tot $B',B''\subseteq B$, si, i només si, $f$ és exhaustiva.}
	\end{legal}
	\item[31.] \textbf{Considerem una aplicació $f: \ A\mapsto B$. Demostreu:}
	\begin{legal}
		\item[31.1.] \textbf{Si $A'\subseteq A$, aleshores $A'\subseteq f^{-1}(f(A'))$.}
		$$f(A')=\{y\in B:(\exists x\in A':f(x)=y)\}.$$
		$$f^{-1}(f(A'))=\{x\in A:f(x)\in f(A')\}.$$
		Tenint en compte que podrien existir elements d'A que corresponguessin amb l'aplicació a elements d'$f(A')$, el conjunt antiimatge $f^{-1}(f(A'))$ és un superconjunt d'$A'$.
		$$\implies A'\subseteq f^{-1}(f(A')).\square$$
		\item[31.2.] \textbf{$f$ és injectiva si i només si $A'=f^{-1}(f(A')) \ \forall A'\subseteq A$.}\\
		Agafant la igualtat que volem demostrar, si apliquem $f$ als dos costats, ens ha de quedar una identitat per poder afirmar que $f$ és injectiva. Com podem efectivament comprovar,
		$$f(A')=f(f^{-1}(f(A')))=Id_B(f(A'))=f(A')$$i $A'=A'$, per tant, queda demostrat l'enunciat.$\square$
	\end{legal}
\end{legal}

\newpage

\section{Relacions, operacions i estructures.}
\begin{legal}
	\item[28.] Considerem a $\mathbb{Z}$ les operacions
	$$a\oplus b=a+b-6;$$
	$$a\odot b=ab+\alpha(a+b)+42,$$
	on $\alpha\in\mathbb{Z}$.
	\begin{legal}
		\item[1)] \textbf{Comproveu que $(\mathbb{Z},\oplus)$ és un grup commutatiu.}\\
		Per ser un grup commutatiu, l'operació $\oplus$ ha de complir les propietats associativa i commutativa i ha de tenir element neutre i invers per tots els elements de $\mathbb{Z}$. Comprovem-ho:
		\begin{itemize}
			\item Associativa: comprovem que $\forall a,b\in\mathbb{Z},\ a\oplus(b\oplus c)=(a\oplus b)\oplus c=a\oplus b\oplus c$.
			$$a\oplus(b\oplus c)=a+(b\oplus c)-6=a+(b+c-6)-6=$$
			$$(a+b-6)+c-6=(a\oplus b)+c-6=(a\oplus b)\oplus c.\square$$
			\item Commutativa: comprovem que $\forall a,b\in\mathbb{Z},\ a\oplus b=b\oplus a$.\\
			$$a\oplus b=a+b-6=b+a-6=b\oplus a.\square$$
			\item Existència del neutre: suposem que $\exists e\in\mathbb{Z}:\ a\oplus e=a$ i el trobem.
			$$a\oplus e=a\iff a+e-6=a\iff e-6=0\iff e=6.$$
			En ser $\oplus$ commutativa, ens podem estalviar comprovar per l'altre costat.
			\item Existència de l'invers: suposem que $\exists a'\in\mathbb{Z}:\ a\oplus a'=e$ i el trobem.
			$$a\oplus a'=e\iff a+a'-6=6\iff a'=12-a.$$
		\end{itemize}
		\item[2)] \textbf{Demostreu que l'operació $\odot$ és associativa si, i només si, $\alpha=-6$ o $\alpha=7$.}\\
		L'operació $\odot$ serà associativa quan $a\odot(b\odot c)=(a\odot b)\odot c=a\odot b\odot c$. Veiem què passa quan igualem les expressions per cada un dels costats de la tesi:
		\begin{itemize}
			\item $a\odot(b\odot c)=a(b\odot c)+\alpha(a+b\odot c)+42=a(bc+\alpha(b+c)+42)+\alpha(a+bc+\alpha(b+c)+42)+42$
			\item $(a\odot b)\odot c=(a\odot b)c+\alpha(a\odot b+c)+42=(ab+\alpha(a+b)+42)c+\alpha(ab+\alpha(a+b)+42+c)+42$
		\end{itemize}
		En igualar,
		$$
		a(bc+\alpha(b+c)+42)+\alpha(a+bc+\alpha(b+c)+42)+42
		$$
		$$
		=(ab+\alpha(a+b)+42)c+\alpha(ab+\alpha(a+b)+42+c)+42.
		$$
		Treiem els $42$ d'ambdós costats i desenvolupem els productes amb la distributivitat del producte sobre la suma usual:
		$$
		abc+a\alpha b+a\alpha c+42a+a\alpha+bc\alpha+\alpha^2b+\alpha^2c+42\alpha=
		$$
		$$
		abc+c\alpha a+c\alpha b+42c+ab\alpha+\alpha^2a+\alpha^2b+42\alpha+c\alpha.
		$$
		Cancel·lant els termes iguals,
		$$
		42a+a\alpha +c\alpha^2=42c+a\alpha^2+c\alpha,
		$$
		i ara reescrivint com una equació de segon grau en $\alpha$ queda:
		$$
		(c-a)\alpha^2+(a-c)\alpha+42(a-c)=0.
		$$
		Aquesta equació la dividim entre $(c-a)$, que és diferent de zero ja que si $a=c$, $(a\odot b)\odot a=a\odot(b\odot a)$ si $\odot$ és commutativa. Ho podem veure ràpid:
		$$a\odot b=ab+\alpha(a+b)+42=ba+\alpha(b+a)+42=b\odot a.$$
		Per tant, com que $a\neq c$, dividim;
		$$
		\alpha^2-\alpha-42=0,
		$$
		i calculem les solucions amb la fórmula pel càlcul de les arrels dels polinomis de segon grau:
		$$
		\alpha=\dfrac{1\pm\sqrt{1-4(-42)}}{2}
		$$
		$$
		\alpha_1=-6\qquad\alpha_2=7.
		$$
		Per tant, seguint el curs de les implicacions i tal com volíem demostrar, $\odot$ només és associativa quan $\alpha=-6$ o $\alpha=7$.$\square$
		\item[3)] \textbf{Demostreu que l'operació $\odot$ té element neutre si, i només si, $\alpha=-6$ o $\alpha=7$.}\\
		Si existeix un element neutre, $\exists e\in\mathbb{Z}:a\odot e=a$. Per tant, ho escrivim:
		$$a\odot e=ae+\alpha(a+e)+42=a.$$
		Si el trobem, haurem demostrat que existeix. Per fer-ho, intentem resoldre l'equació:
		$$a(e+\alpha)+e\alpha+42=a.$$
		Com que estem utilitzant la suma i el producte usuals, la única solució possible es troba solucionant el sistema:
		\[		
		\begin{array}{rcl}
			e+\alpha=1\\
			e\alpha+42=0
		\end{array}
		\]
		D'aquí tenim $e=1-\alpha$ i $(1-\alpha)\alpha+42=0$. Aquesta equació ja la tenim solucionada (apartat 2) i per tant, veiem que $e$ existeix només quan $\alpha=-6$ o $\alpha=7$.$\square$
		\item[4)] \textbf{Per a quins valors de $\alpha$ és $(\mathbb{Z},\oplus,\odot)$ un anell?}\\
		Per a que $(\mathbb{Z},\oplus,\odot)$ sigui un anell necessitem que $(\mathbb{Z},\oplus$ sigui grup abelià i que $\odot$ sigui associativa, tingui element neutre i sigui distributiva respecte $\oplus$. Com ja hem vist, $\oplus$ només és associativa i té neutre quan $\alpha=-6$ o $\alpha=7$. Per ser distributiva, volem que $a\odot(b\oplus c)=(a\odot b)\oplus(a\odot c)$. Ho desenvolupem per ambdós costats:
		\[
		\begin{array}{lcl}
			(1)\quad a\odot(b\oplus c)=a(b\oplus c)+\alpha(a+b\oplus c)+42=ab+ac-6a+\alpha(a+b+c-6)+42,\\
			(2)\quad (a\odot b)\oplus(a\odot c)=(ab+\alpha(a+b)+42)\oplus(ac+\alpha(a+c)+42)=\\
			ab+ac+\alpha(a+b)+\alpha(a+c)+42+42-6.
		\end{array}
		\]
		Ara igualem ambdós resultats i veiem què li ha de passar a $\alpha$:
		$$
		ab+ac-6a+\alpha(a+b+c-6)+42=ab+ac+\alpha(a+b)+\alpha(a+c)+42+42-6
		$$
		$$
		-6a+a\alpha-6\alpha=2a\alpha+36,
		$$
		$$
		-a\alpha-6\alpha=6a+36,
		$$
		$$
		-\alpha(a+6)=6(a+6)\implies \alpha=-6.
		$$
		Per tant, $(\mathbb{Z},\oplus,\odot)$ és un anell per $\alpha=-6$.
	\end{legal}
\end{legal}

\newpage

\section{Conjunts de nombres. Numerabilitat.}

\newpage

\section{El cos dels nombres complexos.}

\newpage

\section{Aritmètica}

\newpage

\section{Polinomis.}
\end{document}
