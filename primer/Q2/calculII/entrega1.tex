\documentclass[11pt]{article}
\usepackage[utf8]{inputenc}
\usepackage{amsfonts}
\usepackage{amsmath,amssymb}
\usepackage{amsthm}
\usepackage{float}
\usepackage[margin=1.25in]{geometry}
\usepackage{color}
\usepackage{breqn}
\title{Càlcul Diferencial}
\author{Àlex Batlle Casellas}
\DeclareMathOperator{\Rn}{\mathbb{R}^n}
\DeclareMathOperator{\Rm}{\mathbb{R}^m}

\begin{document}
\begin{small}
Àlex Batlle Casellas
\end{small}
\paragraph{12.} Si $A\subset\Rm$, demostreu que:
\begin{itemize}
	\item[a)] $\mathring A$ és el conjunt obert més gran contingut en $A$. És a dir, si $B$ és un obert dins $A$, aleshores $B\subset\mathring A$.
	\item[b)] $\bar{A}$ és el conjunt tancat més petit que conté $A$. És a dir, si $C$ és un tancat que conté $A$, aleshores $
\bar{A}\subset C$.
\end{itemize}
\paragraph{Resolució}
\begin{itemize}
	\item[a)] $B$ és obert$\iff\mathring B=B\iff\forall p\in B\ \exists B_r(p)\subset B\implies\forall p\in B\subset A\ \exists B_r(p)\subset B\subset A\implies\mathring B\subset\mathring A\implies B\subset\mathring A.\square$
	\item[b)] $\forall p\in\bar{A}\ (\forall B_r(p)\ A\cap B_r(p)\neq\emptyset)\implies\forall p\in\bar{A}\ (\forall B_r(p)\ C\cap B_r(p)\neq\emptyset)\implies\forall p\in\bar{A},p\in\bar{C}=C\implies\bar{A}\subset C.\square$
\end{itemize}
\paragraph{13.} Donats dos conjunts $A,B$, es defineix $A+B=\{x+y\ \vert\ x\in A, y\in B\}$. Suposeu $A$ obert.
\begin{itemize}
	\item[a)] Demostreu que si $y\in B$, el conjunt $A+\{y\}$ és obert.
	\item[b)] Demostreu que el conjunt $A+B$ és obert.
\end{itemize}
\paragraph{Resolució}
\begin{itemize}
	\item[a)]
	\item[b)]
\end{itemize}
\paragraph{15.} Demostreu que:
\begin{itemize}
	\item[a)] La intersecció d'un nombre arbitrari (finit o infinit) de subconjunts compactes de $\Rn$ també és compacte.
	\item[b)] La unió d'un nombre finit de subconjunts compactes de $\Rn$ també és compacte.
	\item[c)] La unió d'un nombre infinit de subconjunts compactes de $\Rn$ pot no ser compacte. (Doneu-ne exemples).
\end{itemize}
\paragraph{Resolució}
\begin{itemize}
	\item[a)] Com que ens trobem a $\Rn$, n'hi ha prou amb demostrar que la intersecció arbitrària de tancats és tancada i que la intersecció arbitrària de fitats és fitada.
	\begin{itemize}
		\item[Tancada.]
		\item[Fitada.]
	\end{itemize}
	\item[b)]
	\item[c)]
\end{itemize}

\end{document}

