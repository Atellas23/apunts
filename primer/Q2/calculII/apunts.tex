\documentclass[11pt]{article}
\usepackage[utf8]{inputenc}
\usepackage{amsfonts}
\usepackage{amsmath,amssymb}
\usepackage{amsthm}
\usepackage{float}
\usepackage[margin=1.25in]{geometry}
\usepackage{color}
\usepackage{breqn}
\newcommand{\kev}[1]{$\mathbb{#1}$-e.v.}
\newcommand{\undbf}[1]{\underline{\textbf{#1}}:\\}
\newcommand{\propi}{\undbf{Propietats}}
\newcommand{\defi}{\undbf{Definició}}
\newcommand{\prop}{\undbf{Proposició}}
\newcommand{\dem}{\undbf{Demostració}}
\newcommand{\ej}{\underline{Exemples}:\\}
\newcommand{\obs}{\undbf{Observació}}
\newcommand{\cor}{\undbf{Corol·lari}}
\newcommand{\dotpr}[2]{\left< #1,#2\right>}
\newcommand{\norm}[1]{||#1||}
\newcommand{\ball}[2]{B_{#1}(#2)}
\newcommand{\set}[2]{\{#1\ \vert\ #2\}}
\title{Càlcul Diferencial}
\author{Àlex Batlle Casellas}
\renewcommand*\contentsname{Índex}
\DeclareMathOperator{\intr}{Int}
\DeclareMathOperator{\ext}{Ext}
\DeclareMathOperator{\ad}{Ad}
\DeclareMathOperator{\fr}{Fr}

\begin{document}

\begin{titlepage}
	\centering
	{\scshape\LARGE Facultat de Matemàtiques i Estadística \par}
	\vspace{1cm}
	{\scshape\Large Universitat Politècnica de Catalunya - BarcelonaTech\par}
	\vspace{1.5cm}
	{\huge\bfseries Càlcul Diferencial (Q2)
	\par}
	\vspace{2cm}
	{\Large\itshape Àlex Batlle Casellas\par}

	\vfill

% Bottom of the page
	{\large \today\par}
\end{titlepage}

%\section*{Resumen}

\vfill
\newpage\tableofcontents
\newpage
\section{Topologia d'$\mathbb{R}^n$.}
\subsection*{Preliminars.}
\paragraph{Estructura afí de $\mathbb{R}^n$.}
$\mathbb{A}=(A,E,\phi)$ és un espai afí, amb
\begin{align*}
		\phi \colon A\times A &\to E\\
		q &\mapsto \phi(p,q)
\end{align*}
com a funció d'assignació de vectors entre dos punts. Es té que, per $p\in A$ fixat,
\begin{align*}
		\phi_p \colon A &\to E\\
		q &\mapsto \phi(p,q)
\end{align*}
és bijectiva. En aquest curs, prendrem $A=\mathbb{R}^n$.
\subsection{Nocions de topologia.}
\subsubsection{Espais mètrics i normats.}
\defi Sigui $M$ un conjunt. Una \textbf{distància en $M$} és una aplicació
\begin{align*}
		d \colon M\times M &\to \mathbb{R}\\
		(x,y) &\mapsto d(x,y).
\end{align*}
tal que compleix, $\forall x,y,z\in M$:
\begin{enumerate}
	\item Definida positiva: $d(x,y)\geq0$.
	\item No degeneració: $d(x,y)=0\iff x=y$.
	\item Simetria: $d(x,y)=d(y,x)$.
	\item Desigualtat triangular: $d(x,y)\leq d(x,z)+d(y,z)$.
\end{enumerate}
Un \textbf{espai mètric} és un parell $(M,d)$.\\
\defi Sigui $E$ un \kev{R} de dimensió arbitrària. Una \textbf{norma en $E$} és una aplicació
\begin{align*}
		\norm{\cdot} \colon E &\to \mathbb{R}\\
		v &\mapsto \norm{v}
\end{align*}
tal que compleix, $\forall u,v\in E$:
\begin{enumerate}
	\item Definida positiva: $\norm{u}\geq0$.
	\item No degeneració: $\norm{u}=0\iff u=\vec{0}$.
	\item Multiplicació per escalar: $\norm{\lambda u}=|\lambda|\ \norm{u}\forall\lambda\in\mathbb{R}$.
	\item Desigualtat triangular: $\norm{u+v}\leq \norm{u}+\norm{v}$.
\end{enumerate}
Un \textbf{espai normat} és un parell $(E,\norm{\cdot})$.\\
\prop Si $(E,\norm{\cdot})$ és un espai normat, aleshores $(E,d)$ és un espai mètric, amb $d$ la distància associada a la norma $\norm{\cdot}$, $d(u,v):=\norm{u-v}$.\\
\dem Les propietats (1),(2), i (4) d'espai mètric són immediates (s'hereten de les propietats de la norma). Comprovem la propietat (3) d'espai mètric: siguin $u,v\in E$, aleshores
$$
d(u,v):=||u-v||=||-(v-u)||=|-1|\ ||v-u||=||v-u||=:d(v,u).\square
$$
\defi Sigui $E$ un \kev{R} Un \textbf{producte escalar euclidià} en $E$ és una aplicació
\begin{align*}
		\dotpr{\cdot}{\cdot} \colon E\times E &\to \mathbb{R}\\
		(u,v) &\mapsto \dotpr{u}{v}
\end{align*}
tal que compleix, $\forall u,v,w\in E,\alpha,\beta\in\mathbb{R}$:
\begin{enumerate}
	\item Linealitat: $\dotpr{\alpha u+\beta v}{w}=\alpha\dotpr{u}{w}+\beta\dotpr{v}{w}$.
	\item Simetria: $\dotpr{u}{v}=\dotpr{v}{u}$.
	\item Definida positiva: $\dotpr{v}{v}\geq0$.
	\item No degeneració: $\dotpr{v}{v}=0\iff v=\vec{0}$.
\end{enumerate}
Un \textbf{espai euclidià} és un parell $(E,\dotpr{\cdot}{\cdot})$.\\
\prop Si $(E,\dotpr{\cdot}{\cdot})$ és un espai euclidià, aleshores $(E,||\cdot||)$ és un espai normat, amb $||\cdot||$ la norma inuïda per el producte escalar $\left<\cdot,\cdot\right>$, $||u||:=+\sqrt{\left<u,u\right>}$.\\
\dem \\
\prop El producte escalar i la seva norma associada tenen les següents propietats, $\forall u,v\in E$:
\begin{enumerate}
	\item $|\left<\cdot,\cdot\right>|\leq||u||\ ||v||$. (Desigualtat de Cauchy-Schwarz)
	\item $||u-v||\geq||u||-||v||$.
	\item $||u+v||^2+||u-v||^2=2||u||^2+2||v||^2$. (Identitat del paral·lelogram)
	\item $||u+v||^2-||u-v||^2=4\left<u,v\right>$. (Identitat de polarització)
	\item Si $u=(u_i)$, aleshores $|u_i|\leq||u||\leq\sum_{i=1}^n|u_i|$.
\end{enumerate}
\dem \\
\prop A $\mathbb{R}^n$ es defineix:
\begin{enumerate}
	\item $\left<u,v\right>_2:=\sum_{i=1}^nu_iv_i$, on $u=(u_i),v=(v_i)$.
	\item $||u||_2:=\sqrt{\left<u,u\right>_2}=\sqrt{\sum_{i=1}^nu_iv_i}$.
	\item $d(u,v)_2:=||u-v||_2=\sqrt{\left<u-v,u-v\right>_2}=\sqrt{\sum_{i=1}^n(u_i-v_i})^2$.
\end{enumerate}
\dem \\
\subsubsection{Boles i entorns.}
$(M,d)$ espai mètric.\\
\defi $p\in M,r\in\mathbb{R}^+$.
\begin{enumerate}
	\item \textbf{Esfera} amb centre a $p$ i radi $r$:
	$$
	S_r(p)\equiv S(p,r)=\{q\in M\ \vert\ d(p,q)=r\}.
	$$
	\item \textbf{Bola oberta} amb centre a $p$ i radi $r$:
	$$
	\ball{r}{p}\equiv B(p,r)=\{q\in M\ \vert\ d(p,q)<r\}.
	$$
	\item \textbf{Bola tancada} amb centre a $p$ i radi $r$:
	$$
	\bar{B}_r(p)\equiv\bar{B}(p,r)=\{q\in M\ \vert\ d(p,q)\leq r\}.
	$$
	\item \textbf{Bola perforada} amb centre a $p$ i radi $r$ (pot ser oberta o tancada):
	$$
	B^*_r(p)\equiv B^*(p,r)=\{q\in M\ \vert\ d(p,q)\leq r\}-\{p\}.
	$$
\end{enumerate}
\defi $A\subseteq M$. $A$ és un \textbf{conjunt fitat} si $\exists B_r(p),p\in A:\ A\subseteq B_r(p)$.\\
\defi $p\in M$. Un \textbf{entorn de $p$} és un conjunt $E(p)\subseteq M$ fitat tal que $\exists B_r(p)\subseteq E(p)$.
\newpage

\newpage
\end{document}
