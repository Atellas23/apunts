\documentclass[11pt]{article}
\usepackage[utf8]{inputenc}
\usepackage{amsfonts}
\usepackage{amsmath,amssymb}
\usepackage{amsthm}
\usepackage{float}
\usepackage[margin=1.25in]{geometry}
\usepackage{color}
\usepackage{breqn}
\newcommand{\rev}{$\mathbb{R}$-e.v. }
\newcommand{\props}{\underline{\textbf{Propietats}}:\\}
\newcommand{\defi}{\underline{\textbf{Definició}}:\\}
\newcommand{\prop}{\underline{\textbf{Proposició}}:\\}
\newcommand{\dem}{\underline{\textbf{Demostració}}:\\}
\newcommand{\ej}{\underline{Exemples}:\\}
\newcommand{\obs}{\underline{\textbf{Observació}}: }
\newcommand{\cor}{\underline{\textbf{Corol·lari}}:\\}
\newcommand{\scalarproduct}{\left<\cdot,\cdot\right>}
\title{Càlcul Diferencial}
\author{Àlex Batlle Casellas}
\renewcommand*\contentsname{Índex}

\begin{document}

\begin{titlepage}
	\centering
	{\scshape\LARGE Facultat de Matemàtiques i Estadística \par}
	\vspace{1cm}
	{\scshape\Large Universitat Politècnica de Catalunya - BarcelonaTech\par}
	\vspace{1.5cm}
	{\huge\bfseries Càlcul Diferencial (Q2)
	\par}
	\vspace{2cm}
	{\Large\itshape Àlex Batlle Casellas\par}

	\vfill

% Bottom of the page
	{\large \today\par}
\end{titlepage}

%\section*{Resumen}

\vfill
\newpage\tableofcontents
\newpage
\section{Topologia d'$\mathbb{R}^n$.}
\subsection*{Preliminars.}
\paragraph{Estructura afí de $\mathbb{R}^n$.}
$\mathbb{A}=(A,E,\phi)$ és un espai afí, amb
\begin{align*}
		\phi \colon A\times A &\to E\\
		q &\mapsto \phi(p,q)
\end{align*}
com a funció d'assignació de vectors entre dos punts. Es té que, per $p\in A$ fixat,
\begin{align*}
		\phi_p \colon A &\to E\\
		q &\mapsto \phi(p,q)
\end{align*}
és bijectiva. En aquest curs, prendrem $A=\mathbb{R}^n$.
\newpage
\subsection{Nocions de topologia.}
\subsubsection{Espais mètrics i normats.}
\defi Sigui $M$ un conjunt. Una \textbf{distància en $M$} és una aplicació
\begin{align*}
		d \colon M\times M &\to \mathbb{R}\\
		(x,y) &\mapsto d(x,y).
\end{align*}
tal que compleix, $\forall x,y,z\in M$:
\begin{enumerate}
	\item Definida positiva: $d(x,y)\geq0$.
	\item No degeneració: $d(x,y)=0\iff x=y$.
	\item Simetria: $d(x,y)=d(y,x)$.
	\item Desigualtat triangular: $d(x,y)\leq d(x,z)+d(y,z)$.
\end{enumerate}
Un \textbf{espai mètric} és un parell $(M,d)$.\\
\defi Sigui $E$ un \rev de dimensió arbitrària. Una \textbf{norma en $E$} és una aplicació
\begin{align*}
		||\cdot|| \colon E &\to \mathbb{R}\\
		v &\mapsto ||v||
\end{align*}
tal que compleix, $\forall u,v\in E$:
\begin{enumerate}
	\item Definida positiva: $||u||\geq0$.
	\item No degeneració: $||u||=0\iff u=\vec{0}$.
	\item Multiplicació per escalar: $||\lambda u||=|\lambda|\ ||u||\forall\lambda\in\mathbb{R}$.
	\item Desigualtat triangular: $||u+v||\leq ||u||+||v||$.
\end{enumerate}
Un \textbf{espai normat} és un parell $(E,||\cdot||)$.\\
\prop Si $(E,||\cdot||)$ és un espai normat, aleshores $(E,d)$ és un espai mètric, amb $d$ la distància associada a la norma $||\cdot||$, $d(u,v):=||u-v||$.\\
\dem Les propietats (1),(2), i (4) d'espai mètric són immediates (s'hereten de les propietats de la norma). Comprovem la propietat (3) d'espai mètric: siguin $u,v\in E$, aleshores
$$
d(u,v):=||u-v||=||-(v-u)||=|-1|\ ||v-u||=||v-u||=:d(v,u).\square
$$
\defi Sigui $E$ un \rev Un \textbf{producte escalar euclidià} en $E$ és una aplicació
\begin{align*}
		\left<\cdot,\cdot\right> \colon E\times E &\to \mathbb{R}\\
		(u,v) &\mapsto \left< u,v\right>
\end{align*}
tal que compleix, $\forall u,v,w\in E,\alpha,\beta\in\mathbb{R}$:
\begin{enumerate}
	\item Linealitat: $\left<\alpha u+\beta v,w\right>=\alpha\left<u,w\right>+\beta\left<v,w\right>$.
	\item Simetria: $\left<u,v\right>=\left<v,u\right>$.
	\item Definida positiva: $\left<v,v\right>\geq0$.
	\item No degeneració: $\left<v,v\right>=0\iff v=\vec{0}$.
\end{enumerate}
Un \textbf{espai euclidià} és un parell $(E,\left<\cdot,\cdot\right>)$.\\
\prop Si $(E,\left<\cdot,\cdot\right>)$ és un espai euclidià, aleshores $(E,||\cdot||)$ és un espai normat, amb $||\cdot||$ la norma inuïda per el producte escalar $\left<\cdot,\cdot\right>$, $||u||:=+\sqrt{\left<u,u\right>}$.\\
\dem \\
\prop El producte escalar i la seva norma associada tenen les següents propietats, $\forall u,v\in E$:
\begin{enumerate}
	\item $|\left<\cdot,\cdot\right>|\leq||u||\ ||v||$. (Desigualtat de Cauchy-Schwarz)
	\item $||u-v||\geq||u||-||v||$.
	\item $||u+v||^2+||u-v||^2=2||u||^2+2||v||^2$. (Identitat del paral·lelogram)
	\item $||u+v||^2-||u-v||^2=4\left<u,v\right>$. (Identitat de polarització)
	\item Si $u=(u_i)$, aleshores $|u_i|\leq||u||\leq\sum_{i=1}^n|u_i|$.
\end{enumerate}
\dem \\
\prop A $\mathbb{R}^n$ es defineix:
\begin{enumerate}
	\item $\left<u,v\right>_2:=\sum_{i=1}^nu_iv_i$, on $u=(u_i),v=(v_i)$.
	\item $||u||_2:=\sqrt{\left<u,u\right>_2}=\sqrt{\sum_{i=1}^nu_iv_i}$.
	\item $d(u,v)_2:=||u-v||_2=\sqrt{\left<u-v,u-v\right>_2}=\sqrt{\sum_{i=1}^n(u_i-v_i})^2$.
\end{enumerate}
\dem \\
\subsubsection{Boles i entorns.}
$(M,d)$ espai mètric.\\
\defi $p\in M,r\in\mathbb{R}^+$.
\begin{enumerate}
	\item \textbf{Esfera} amb centre a $p$ i radi $r$:
	$$
	S_r(p)\equiv S(p,r)=\{q\in M\ \vert\ d(p,q)=r\}.
	$$
	\item \textbf{Bola oberta} amb centre a $p$ i radi $r$:
	$$
	B_r(p)\equiv B(p,r)=\{q\in M\ \vert\ d(p,q)<r\}.
	$$
	\item \textbf{Bola tancada} amb centre a $p$ i radi $r$:
	$$
	\bar{B}_r(p)\equiv\bar{B}(p,r)=\{q\in M\ \vert\ d(p,q)\leq r\}.
	$$
	\item \textbf{Bola perforada} amb centre a $p$ i radi $r$ (pot ser oberta o tancada):
	$$
	B^*_r(p)\equiv B^*(p,r)=\{q\in M\ \vert\ d(p,q)\leq r\}-\{p\}.
	$$
\end{enumerate}
\defi $A\subseteq M$. $A$ és un \textbf{conjunt fitat} si $\exists B_r(p),p\in A:\ A\subseteq B_r(p)$.\\
\defi $p\in M$. Un \textbf{entorn de $p$} és un conjunt $E(p)\subseteq M$ fitat tal que $\exists B_r(p)\subseteq E(p)$.
\newpage

\newpage
\end{document}
