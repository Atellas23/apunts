\documentclass[10pt]{article}
\usepackage[utf8]{inputenc}
\usepackage{amsfonts}
\usepackage{amsmath,amssymb}
\usepackage{amsthm}
\usepackage{float}
\usepackage[margin=1.25in]{geometry}
\usepackage{color}
\usepackage{breqn}
\newcommand{\af}{\mathbb{A}}
\newcommand{\euc}[1]{\mathbb{E}_{\mathbb{R}}^{#1}}
\newcommand{\norm}[1]{||#1||}
\newcommand{\scal}[2]{\left<#1,#2\right>}
\newcommand{\obs}{\underline{\textbf{Observació}}: }
\DeclareMathOperator{\idn}{Id}
\author{Àlex Batlle Casellas}

\begin{document}
\begin{small}
Àlex Batlle Casellas
\end{small}\\
\paragraph{1.}
\begin{itemize}
	\item[(i)] Sigui $F$ un gir de $\euc{2}$ al voltant d'un punt $p$. Si $F(q_1)=q_2$, demostreu que $p$ pertany a la recta perpendicular a $u=q_2-q_1$ i que conté al punt $(q_1+q_2)/2$. Com a aplicació, trobeu el gir de $\euc{2}$ que verifica $F(1,1)=(-1,3)$ i que $F(2,0)=(0,4)$ (equacions, centre i angle de gir).
	\item[(ii)] Sigui $F$ una rotació al voltant d'una recta $r$ de $\euc{3}$. Si $F(p)=q$, demostreu que l'eix $r$ pertany al pla $\pi$ perpendicular al vector $u=q-p$ i que conté al punt $(p+q)/2$. Com a aplicació,si $F$ és una rotació de $\euc{3}$ tal que $F(1,1,1)=(1,1,0)$ i que $F(0,1,0)=(1,0,1)$, trobeu l'eix, l'angle de gir i les equacions de $F$. 
\end{itemize}
\textbf{Resolució}\\
\begin{itemize}
\item[(i)] Podem resoldre la primera part d'aquest apartat passant per la definició de mediatriu. La mediatriu (a $\euc{2}$) de dos punts és el lloc geomètric de tots els punts que equidisten dels dos donats. Aleshores, primer notem que tant $p$ com el punt mig $M:=(q_1+q_2)/2$ pertanyen a la mediatriu. Això es justifica pel punt $p$ de la següent manera: com que $F$ és un moviment, la distància entre dos punts i les seves imatges es conserva. Per tant, tenim que $\norm{q_1-p}=\norm{F(q_1)-F(p)}=\norm{q_2-p}$ ($p$ és punt fix per definició). Com a conseqüència d'aquest fet, $p$ pertany a la mediatriu. Pel punt $M$ no cal justificar que hi pertany, ja que per definició de punt mig ja equidista de $q_1$ i $q_2$. Ara volem veure que la mediatriu és perpendicular a la recta $\left<q_1,q_2\right>$. Per tant, sigui $x$ un punt de la mediatriu. Aleshores, tenim la següent cadena d'equivalències (abús de notació: s'ometran les fletxes indicadores de vector, és a dir, que $\vec{AB}\equiv AB$):
    \begin{multline}
	\norm{q_1x}=\norm{q_2x}\iff\scal{q_1x}{q_1x}=\scal{q_2x}{q_2x}\iff\scal{q_1M+Mx}{q_1M+Mx}= \\ \scal{q_2M+Mx}{q_2M+Mx}\iff\scal{q_1M}{q_1M}+\scal{Mx}{q_1M}+\scal{q_1M}{Mx}+\scal{Mx}{Mx}= \\ \scal{q_2M}{q_2M}+\scal{Mx}{q_2M}+\scal{q_2M}{Mx}+\scal{Mx}{Mx}\iff\scal{q_1M}{Mx}=\scal{q_2M}{Mx}\iff \\ \scal{q_1M-q_2M}{Mx}=0\iff\scal{q_1M+Mq_2}{Mx}=0\iff\scal{q_1q_2}{Mx}=0.\square
    \end{multline}
    Per tant, la mediatriu és perpendicular a la recta $\scal{q_1}{q_2}$.\\
    Pel cas particular donat, per trobar les equacions, angle i centre de gir, comencem plantejant com ha de ser la matriu d'aquest gir: com que estem a $\euc{2}$, la matriu de la part lineal de $F$ ha de ser (en alguna base) de la forma
    \[
    \begin{pmatrix}
    \cos{\alpha} & -\sin{\alpha}\\
    \sin{\alpha} & \cos{\alpha}
    \end{pmatrix}.
    \]
    Per tant, plantegem les següents equacions:
    \begin{align}
        \begin{pmatrix}
        \cos{\alpha} & -\sin{\alpha}\\
        \sin{\alpha} & \cos{\alpha}
        \end{pmatrix}\begin{pmatrix}1\\ 1
        \end{pmatrix}+\begin{pmatrix}m\\ n
        \end{pmatrix}=\begin{pmatrix}-1\\ 3
        \end{pmatrix}\\
        \begin{pmatrix}
        \cos{\alpha} & -\sin{\alpha}\\
        \sin{\alpha} & \cos{\alpha}
        \end{pmatrix}\begin{pmatrix}2\\ 0
        \end{pmatrix}+\begin{pmatrix}m\\ n
        \end{pmatrix}=\begin{pmatrix}0\\ 4
        \end{pmatrix}.
    \end{align}
    Si restem la primera a la segona, queda la següent equació,
    \[
        \begin{pmatrix}
        \cos{\alpha} & -\sin{\alpha}\\
        \sin{\alpha} & \cos{\alpha}
        \end{pmatrix}\begin{pmatrix}1\\ -1
        \end{pmatrix}=\begin{pmatrix}1\\ 1
        \end{pmatrix},
    \]
    que té per solució $\cos{\alpha}=0,\sin{\alpha}=1$; d'aquí tenim l'angle de gir, $\alpha=\dfrac{\pi}{2}$. Per trobar el terme independent de les equacions de $F$, agafem qualsevol de les dues equacions anteriors; per exemple, (2):
    \[    
        \begin{pmatrix}
        0 & -1\\
        1 & 0
        \end{pmatrix}\begin{pmatrix}1\\ 1
        \end{pmatrix}+\begin{pmatrix}m\\ n
        \end{pmatrix}=\begin{pmatrix}-1\\ 3
        \end{pmatrix}\implies\begin{pmatrix}m\\ n
        \end{pmatrix}=\begin{pmatrix}0\\ 2
        \end{pmatrix}.
    \]
    Finalment, busquem el centre de gir, un punt fix:
    \[
    \begin{pmatrix}
    0 & -1\\
    1 & 0
    \end{pmatrix}\begin{pmatrix}x_0\\ y_0
    \end{pmatrix}+\begin{pmatrix}0\\ 2
    \end{pmatrix}=\begin{pmatrix}x_0\\ y_0
    \end{pmatrix}\implies\begin{pmatrix}x_0\\ y_0
    \end{pmatrix}=\begin{pmatrix}-1\\ 1
    \end{pmatrix}.
    \]
    Per tant, ja tenim totes les dades del moviment; es tracta d'un gir d'angle $\alpha=\dfrac{\pi}{2}$ i de centre $p_0=\begin{pmatrix}-1\\ 1\end{pmatrix}$. Les seves equacions són:
    \[
    F\begin{pmatrix}x\\ y\end{pmatrix}=\begin{pmatrix}
    0 & -1\\
    1 & 0
    \end{pmatrix}\begin{pmatrix}x\\ y
    \end{pmatrix}+\begin{pmatrix}0\\ 2
    \end{pmatrix}.
    \]
\item[(ii)] Per la primera part d'aquest apartat, fem un raonament semblant a l'anterior: com que les distàncies $d(p,r)$ i $d(F(q),F(r))=d(p,r)$ es mantenen iguals a través de la rotació (perquè és un moviment), l'eix $r$ pertany al pla mediatriu de $p$ i $q$. Utilitzant la demostració (1) amb $x$ un punt del pla mediatriu, podem veure altre cop que aquest és efectivament perpendicular a la recta $\scal{p}{q}$ i per la definició de mediatriu, conté el punt mig del segment $\bar{pq}$.\\
Per trobar les equacions de la $F$ en particular que ens demanen a la segona part, farem el següent: construïrem els plans perpendiculars a cada segment entre els punts donats i les respectives imatges, en farem la intersecció, i trobarem l'eix al voltant del qual es rota. L'angle de gir el calcularem directament un cop tinguem l'eix. I les equacions de $F$ les escriurem en una referència ortonormal $\bar{\mathcal{R}}=\{p_0;v_1,v_2,v_3\}$ adequada, en la qual tindran la forma
\[
F\begin{pmatrix}
\bar{x}\\
\bar{y}\\
\bar{z}\\
1
\end{pmatrix}=\begin{pmatrix}
1 & 0 & 0 & 0\\
0 & \cos{\alpha} & -\sin{\alpha} & 0\\
0 & \sin{\alpha} & \cos{\alpha} & 0\\
0 & 0 & 0 & 1
\end{pmatrix}\begin{pmatrix}
\bar{x}\\
\bar{y}\\
\bar{z}\\
1
\end{pmatrix}.
\]
Aleshores, siguin $p_1=(1,1,1),F(p_1)=q_1=(1,1,0)$ i $p_2=(0,1,0),F(p_2)=q_2=(1,0,1)$; el vector $u_1=\vec{p_1q_1}=\begin{pmatrix}0\\ 0\\ -1\end{pmatrix}$, i el vector $u_2=\vec{p_2q_2}=\begin{pmatrix}1\\ -1\\ 1\end{pmatrix}$; i els punts mitjos de cada segment, $M_1=\dfrac{1}{2}(p_1+q_1)=\left(1,1,\dfrac{1}{2}\right),M_2=\dfrac{1}{2}(p_2+q_2)=\left(\dfrac{1}{2},\dfrac{1}{2},\dfrac{1}{2}\right)$. Amb aquesta informació, construïm els plans perpendiculars a cada segment que passen pels respectius punts mitjos, dels que forma part l'eix (com hem demostrat) i per tant, en fer-ne la intersecció tenim l'eix de rotació. $\pi_i$ és el pla perpendicular al segment $\bar{p_iq_i}$ que passa pel seu punt mig $M_i$ ($i=1,2$). Per tant, surten
\[
\begin{cases}
\pi_1:\quad z=\dfrac{1}{2};\\
\pi_2:\quad -x+y-z=\dfrac{1}{2},
\end{cases}
\]
i la intersecció dóna l'eix de rotació, $r=\left(0,1,\dfrac{1}{2}\right)+\left[\begin{pmatrix}
1\\ 1\\ 0
\end{pmatrix}\right]$.
Ara calculem el pla perpendicular a l'eix de rotació en el que es trobin, per exemple, $p_1$ i $q_1$. En general, el pla perpendicular a l'eix tindrà equació (depenent del punt de l'eix pel que passi)
\begin{equation}
\pi_d:\quad x+y=d.
\end{equation}
El pla que busquem passa per $p_1$ i $q_1$; cal observar que amb que passi per un dels dos ja fem. Per tant, substituïm els valors de les coordenades de $p_1$ a (4), l'equació de $\pi_d$ i trobem el pla perpendicular a $r$ al qual pertanyen $p_1$ i $q_1$; l'anomenarem $\pi_{\perp}$. Per tant, ens queda
\begin{equation}
\pi_{\perp}:\quad x+y=2.
\end{equation}
Calculem la intersecció amb l'eix de rotació; surt el punt $o_1=\left(\dfrac{1}{4},\dfrac{3}{2},\dfrac{1}{2}\right)$. Ara, mirem l'angle $\alpha$ que hi ha entre els vectors $v_p=\vec{o_1p_1}$ i $v_q=\vec{o_1q_1}$:
\[
\cos{\alpha}=\dfrac{\scal{v_p}{v_q}}{\norm{v_p}\norm{v_q}}=\dfrac{1}{4}.
\]
Si orientem l'eix pel vector $\begin{pmatrix}
1\\ 1\\ 0
\end{pmatrix}$, l'angle de gir queda a l'interval $(0,\pi]$. Per tant, queda $\alpha\approx 1.31812\text{ rad}$, i les equacions de $F$ queden, en la referència especial $\bar{\mathcal{R}}$,
\[
F\begin{pmatrix}
\bar{x}\\
\bar{y}\\
\bar{z}\\
1
\end{pmatrix}=\begin{pmatrix}
1 & 0 & 0 & 0\\
0 & \dfrac{1}{4} & -\dfrac{\sqrt{15}}{4} & 0\\
0 & \dfrac{\sqrt{15}}{4} & \dfrac{1}{4} & 0\\
0 & 0 & 0 & 1
\end{pmatrix}\begin{pmatrix}
\bar{x}\\
\bar{y}\\
\bar{z}\\
1
\end{pmatrix}.
\]
\end{itemize}
\paragraph{2.} A l'espai euclidià $\euc{3}$ amb la referència canònica, considerem els moviments $f,g,h$ següents: $f$ i $g$ són les simetries especulars respecte els plants $\pi:\ x-y=0$ i $\pi':\ x+y+z=0$ respectivament, i $h$ té expressió
\[
h(x,y,z)=\dfrac{1}{3}(x-2y-2z+6,-2x-2y+z+6,-2x+y-2z+6).
\]
\begin{itemize}
	\item[(i)] Classifiqueu el moviment $F=h\circ f$, tot donant els seus elements característics.
	\item[(ii)] Classifiqueu i doneu els elements característics de $G=g\circ F$. (Indicació: no cal trobar l'expressió explícita de $G$).
	\item[(iii)] Calculeu $F^{15}(0,0,0)$ i $G^{18}(0,0,1)$.
\end{itemize}
\textbf{Resolució}\\
Informació extreta de l'enunciat: les matrius d'$f$ i $g$ en referència canònica són
	\[
	M(f;\mathcal{R}_{\text{ord}})=\begin{pmatrix}
	0 & 1 & 0 & 0\\
	1 & 0 & 0 & 0\\
	0 & 0 & 1 & 0\\
	0 & 0 & 0 & 1
	\end{pmatrix};\quad M(g;\mathcal{R}_{\text{ord}})=\begin{pmatrix}
	-1 & 0 & 0 & 0\\
	0 & -1 & 0 & 0\\
	0 & 0 & -1 & 0\\
	0 & 0 & 0 & 1
	\end{pmatrix}.
	\]
\begin{itemize}
	\item[(i)] Primer cal posar $h$ en forma matricial, que queda
	\[
	h(x,y,z)=\begin{pmatrix}
	1/3 & -2/3 & -2/3\\
	-2/3 & -2/3 & 1/3\\
	-2/3 & 1/3 & -2/3\\
	\end{pmatrix}
	\begin{pmatrix}
	x\\ y\\ z
	\end{pmatrix}
	+\begin{pmatrix}
	6\\ 6\\ 6
	\end{pmatrix}.
	\]
	D'aquí podem extreure tant si $h$ és directe com si té punts fixos; veiem que és un moviment directe, i que té una recta de punts fixos, $r=(9,0,0)+[(-2,1,1)]$. A més, pel teorema de classificació podem igualar les traces de la matriu d'$h$ en dues bases diferents, el que ens diu que $h$ és una rotació de $\pi$ rad al voltant de $r$. Sabem que $f$ és un moviment invers, i per tant $F$ és un moviment invers. Per saber si $F$ té punts fixos, agafem un subespai director de $\pi$ i veiem si el vector director de $r$ hi pertany. Per exemple, un subespai director de $\pi$ és $[(1,1,0),(0,0,1)]$. Com que el vector director de $r$ no hi pertany, $r\cap\pi$ és un sol punt, i pel teorema de classificació, determinem que $F$ es tracta d'un moviment invers amb un sol punt fix, que calculem tot seguit (calculant la intersecció entre $r$ i $\pi$); surt $p_0=(3,3,3)$.
	\item[(ii)] Sabem que $G$ és un moviment directe, ja que $\det{\tilde{G}}=\det{\tilde{g}}\det{\tilde{F}}$, i per tant $\det{\tilde{G}}=1$, i que no té punts fixos: en particular, l'únic punt fix de $F$ no pertany al pla de punts fixos de $g$. Per tant, $G$ és un moviment helicoidal. Per calcular el vector de translació característic de $G$, agafarem el punt fix de $F$ i li aplicarem $G$. $G(3,3,3)=(-3,-3,-3)$, i per tant el vector de translació ha de ser $v=(-6,-6,-6)$. L'angle de gir el trobarem calculant la matriu de la part lineal de $G$ i aleshores, igualant la traça amb la forma del teorema de classificació; sortirà $\alpha=\dfrac{\pi}{3}$.
\end{itemize}

\end{document}