\documentclass[10pt]{article}
\usepackage[utf8]{inputenc}
\usepackage{amsfonts}
\usepackage{amsmath,amssymb}
\usepackage{amsthm}
\usepackage{float}
\usepackage[margin=1.25in]{geometry}
\usepackage{color}
\usepackage{breqn}
\newcommand{\af}{\mathbb{A}}
\newcommand{\euc}[1]{\mathbb{E}_{\mathbb{R}}^{#1}}
\newcommand{\norm}[1]{||#1||}
\newcommand{\scal}[2]{\left<#1,#2\right>}
\newcommand{\obs}{\underline{\textbf{Observació}}: }
\DeclareMathOperator{\idn}{Id}
\author{Àlex Batlle Casellas}

\begin{document}
\begin{small}
Àlex Batlle Casellas
\end{small}\\
\paragraph{1.}
\begin{itemize}
	\item[(i)] Sigui $F$ un gir de $\euc{2}$ al voltant d'un punt $p$. Si $F(q_1)=q_2$, demostreu que $p$ pertany a la recta perpendicular a $u=q_2-q_1$ i que conté al punt $(q_1+q_2)/2$. Com a aplicació, trobeu el gir de $\euc{2}$ que verifica $F(1,1)=(-1,3)$ i que $F(2,0)=(0,4)$ (equacions, centre i angle de gir).
	\item[(ii)] Sigui $F$ una rotació al voltant d'una recta $r$ de $\euc{3}$. Si $F(p)=q$, demostreu que l'eix $r$ pertany al pla $\pi$ perpendicular al vector $u=q-p$ i que conté al punt $(p+q)/2$. Com a aplicació,si $F$ és una rotació de $\euc{3}$ tal que $F(1,1,1)=(1,1,0)$ i que $F(0,1,0)=(1,0,1)$, trobeu l'eix, l'angle de gir i les equacions de $F$. 
\end{itemize}
\textbf{Resolució}\\
\begin{itemize}
\item[(i)] Podem resoldre la primera part d'aquest apartat passant per la definició de mediatriu. La mediatriu (a $\euc{2}$) de dos punts és el lloc geomètric de tots els punts que equidisten dels dos donats. Aleshores, primer notem que tant $p$ com el punt mig $M:=(q_1+q_2)/2$ pertanyen a la mediatriu. Això es justifica pel punt $p$ de la següent manera: com que $F$ és un moviment, la distància entre dos punts i les seves imatges es conserva. Per tant, tenim que $\norm{q_1-p}=\norm{F(q_1)-F(p)}=\norm{q_2-p}$ ($p$ és punt fix per definició). Com a conseqüència d'aquest fet, $p$ pertany a la mediatriu. Pel punt $M$ no cal justificar que hi pertany, ja que per definició de punt mig ja equidista de $q_1$ i $q_2$. Ara volem veure que la mediatriu és perpendicular a la recta $\left<q_1,q_2\right>$. Per tant, sigui $x$ un punt de la mediatriu. Aleshores, tenim la següent cadena d'equivalències (abús de notació: s'ometran les fletxes indicadores de vector, és a dir, que $\vec{AB}\equiv AB$):
    \begin{multline*}
	\norm{q_1x}=\norm{q_2x}\iff\scal{q_1x}{q_1x}=\scal{q_2x}{q_2x}\iff\scal{q_1M+Mx}{q_1M+Mx}= \\ \scal{q_2M+Mx}{q_2M+Mx}\iff\scal{q_1M}{q_1M}+\scal{Mx}{q_1M}+\scal{q_1M}{Mx}+\scal{Mx}{Mx}= \\ \scal{q_2M}{q_2M}+\scal{Mx}{q_2M}+\scal{q_2M}{Mx}+\scal{Mx}{Mx}\iff\scal{q_1M}{Mx}=\scal{q_2M}{Mx}\iff \\ \scal{q_1M-q_2M}{Mx}=0\iff\scal{q_1M+Mq_2}{Mx}=0\iff\scal{q_1q_2}{Mx}=0.\square
    \end{multline*}
    Per tant, la mediatriu és perpendicular a la recta $\scal{q_1}{q_2}$.\\
    Pel cas particular donat, per trobar les equacions, angle i centre de gir, comencem plantejant com ha de ser la matriu d'aquest gir: com que estem a $\euc{2}$, la matriu de la part lineal de $F$ ha de ser (en alguna base) de la forma
    \[
    \begin{pmatrix}
    \cos{\alpha} & -\sin{\alpha}\\
    \sin{\alpha} & \cos{\alpha}
    \end{pmatrix}.
    \]
    Per tant, plantegem les següents equacions:
    \begin{align}
        \begin{pmatrix}
        \cos{\alpha} & -\sin{\alpha}\\
        \sin{\alpha} & \cos{\alpha}
        \end{pmatrix}\begin{pmatrix}1\\ 1
        \end{pmatrix}+\begin{pmatrix}m\\ n
        \end{pmatrix}=\begin{pmatrix}-1\\ 3
        \end{pmatrix}\\
        \begin{pmatrix}
        \cos{\alpha} & -\sin{\alpha}\\
        \sin{\alpha} & \cos{\alpha}
        \end{pmatrix}\begin{pmatrix}2\\ 0
        \end{pmatrix}+\begin{pmatrix}m\\ n
        \end{pmatrix}=\begin{pmatrix}0\\ 4
        \end{pmatrix}.
    \end{align}
    Si restem la primera a la segona, queda la següent equació,
    \[
        \begin{pmatrix}
        \cos{\alpha} & -\sin{\alpha}\\
        \sin{\alpha} & \cos{\alpha}
        \end{pmatrix}\begin{pmatrix}1\\ -1
        \end{pmatrix}=\begin{pmatrix}1\\ 1
        \end{pmatrix},
    \]
    que té per solució $\cos{\alpha}=0,\sin{\alpha}=1$; d'aquí tenim l'angle de gir, $\alpha=\dfrac{\pi}{2}$. Per trobar el terme independent de les equacions de $F$, agafem qualsevol de les dues equacions anteriors; per exemple, (1):
    \[    
        \begin{pmatrix}
        0 & -1\\
        1 & 0
        \end{pmatrix}\begin{pmatrix}1\\ 1
        \end{pmatrix}+\begin{pmatrix}m\\ n
        \end{pmatrix}=\begin{pmatrix}-1\\ 3
        \end{pmatrix}\implies\begin{pmatrix}m\\ n
        \end{pmatrix}=\begin{pmatrix}0\\ 2
        \end{pmatrix}.
    \]
    Finalment, busquem el centre de gir, un punt fix:
    \[
    \begin{pmatrix}
    0 & -1\\
    1 & 0
    \end{pmatrix}\begin{pmatrix}x_0\\ y_0
    \end{pmatrix}+\begin{pmatrix}0\\ 2
    \end{pmatrix}=\begin{pmatrix}x_0\\ y_0
    \end{pmatrix}\implies\begin{pmatrix}x_0\\ y_0
    \end{pmatrix}=\begin{pmatrix}-1\\ 1
    \end{pmatrix}.
    \]
    Per tant, ja tenim totes les dades del moviment; es tracta d'un gir d'angle $\alpha=\dfrac{\pi}{2}$ i de centre $p_0=\begin{pmatrix}-1\\ 1\end{pmatrix}$. Les seves equacions són:
    \[
    F\begin{pmatrix}x\\ y\end{pmatrix}=\begin{pmatrix}
    0 & -1\\
    1 & 0
    \end{pmatrix}\begin{pmatrix}x\\ y
    \end{pmatrix}+\begin{pmatrix}0\\ 2
    \end{pmatrix}.
    \]
\item[(ii)]
\end{itemize}
\end{document}