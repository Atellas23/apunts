\documentclass[11pt]{article}
\usepackage[utf8]{inputenc}
\usepackage{amsfonts}
\usepackage{amsmath,amssymb}
\usepackage{amsthm}
\usepackage{float}
\usepackage[margin=1.25in]{geometry}
\usepackage{color}
\usepackage{breqn}
\newcommand{\af}{\mathbb{A}}
\newcommand{\obs}{\underline{\textbf{Observació}}: }
\DeclareMathOperator{\nuc}{Nuc}
\DeclareMathOperator{\idn}{Id}
\author{Àlex Batlle Casellas}

\begin{document}
\begin{small}
Àlex Batlle Casellas
\end{small}\\
\paragraph{1.}	A la família d'afinitats de $\af^2_{\mathbb{R}}$ d'equacions
\[
\begin{cases}
	x'=ax+ay+b\\
	y'=ax+6y+b^2
\end{cases}
\]
hi ha quatre homologies els eixos de les quals són els costats d'un paral·lelogram. Determineu els vèrtexs d'aquest paral·lelogram.\\
\textbf{Resolució}\\
Podem expressar les afinitats corresponents com
\[
f_{a,b}(x,y)=\begin{pmatrix} a & a\\ a & 6\end{pmatrix}\begin{pmatrix} x\\ y\end{pmatrix}+\begin{pmatrix} b \\ b^2\end{pmatrix}=M\begin{pmatrix} x\\ y\end{pmatrix}+\hat{b}.
\]
Aleshores, els punts fixos compleixen $(M-\idn)\begin{pmatrix}x\\ y\end{pmatrix}+\hat{b}=\begin{pmatrix}0\\ 0\end{pmatrix}.$ Si escrivim això com a sistema lineal, trobarem valors d'$a$ i $b$ pels quals tenim rectes de punts fixos:
\[
\left(\begin{array}{cc|c}
a-1&a&-b\\
a&5&-b^2\\
\end{array}\right)\sim
\left(\begin{array}{cc|c}
-1&a-5&-b+b^2\\
0&a^2-5a+5&(a-1)b^2-ab\\
\end{array}\right)
\]
Si volem que aquest sistema tingui per solució una recta, la matriu ha de tenir rang 1, i per tant, la segona fila ha de ser tota de zeros. Per tant, resolem les equacions que ens surten d'igualar els elements de la segona fila a zero, és a dir,
\[
a^2-5a+5=0,\qquad (a-1)b^2-ab=0,
\]
de les que surten les solucions $a=\dfrac{5\pm\sqrt{5}}{2},b=0,\dfrac{5\pm\sqrt{5}}{3\pm\sqrt{5}}.$ En imposar aquests valors, podem agafar-ne exactament quatre combinacions, que fan una recta cada una.
\[
\begin{aligned}
r_1:\ -x+\dfrac{-5+\sqrt{5}}{2}y=0,\\
r_2:\ -x+\dfrac{-5-\sqrt{5}}{2}y=0,\\
r_3:\ -x+\dfrac{-5+\sqrt{5}}{2}y=5-2\sqrt{5},\\
r_4:\ -x+\dfrac{-5-\sqrt{5}}{2}y=5+2\sqrt{5}.
\end{aligned}
\]
Ara busquem les interseccions de les rectes que no són paral·leles i haurem trobat els vèrtexs del paral·lelogram.
\begin{itemize}
	\item $r_1\cap r_2:x=0,y=0.$
	\item $r_1\cap r_4:\begin{cases}
	-x+\dfrac{-5+\sqrt{5}}{2}y=0\\
	-x+\dfrac{-5-\sqrt{5}}{2}y=5+2\sqrt{5}
	\end{cases}\implies(\cdots)\implies x=\dfrac{3\sqrt{5}+5}{2},y=-\sqrt{5}-2.$
	\item $r_3\cap r_2:\begin{cases}
	-x+\dfrac{-5-\sqrt{5}}{2}y=0\\
	-x+\dfrac{-5+\sqrt{5}}{2}y=5-2\sqrt{5}
	\end{cases}\implies(\cdots)\implies x=\dfrac{-3\sqrt{5}+5}{2},y=\sqrt{5}-2$
	\item $r_3\cap r_4:\begin{cases}
	-x+\dfrac{-5+\sqrt{5}}{2}y=5-2\sqrt{5}\\
	-x+\dfrac{-5-\sqrt{5}}{2}y=5+2\sqrt{5}
	\end{cases}\implies(\cdots)\implies x=-5,y=-4.$
\end{itemize}
Per tant, ja hem trobat els quatre vèrtexs, que són
\begin{itemize}
	\item[$A$]$=(0,0)$
	\item[$B$]$=(\dfrac{3\sqrt{5}+5}{2},-\sqrt{5}-2)$
	\item[$C$]$=(\dfrac{-3\sqrt{5}+5}{2},\sqrt{5}-2)$
	\item[$D$]$=(-5,-4).$
\end{itemize}
\newpage

\paragraph{2.}	Siguin $F,A,B$ tres punts alineats del pla afí. Considerem totes les afinitats del pla que deixen fix $F$, transformen $A$ en $B$ i tenen una única recta fixa.
\begin{enumerate}
	\item[(a)] Trobeu el lloc geomètric de les imatges d'un punt donat per aquestes afinitats.
	\item[(b)] Demostreu que existeix una homotècia tal que les afinitats anteriors són el producte d'aquesta homotècia per les homologies especials d'eix $FA$.
\end{enumerate}
\textbf{Resolució}\\
Primer de tot, expressarem les afinitats que ens demanen en equacions amb un sistema de referència $\mathcal{R}_0=\{F;v_1,\vec{FA}\}$, agafant $v_1$ qualsevol vector que compleixi $(\tilde{f}-\lambda\idn)(v_1)=\vec{FA}$. Per les hipòtesis del problema, $(F,A,B)=\lambda$. Aleshores, tenim la matriu de l'afinitat en coordenades ampliades de la següent forma:
\[M(f_{\lambda};\mathcal{R}_0)=
\left(\begin{array}{cc|c}
\lambda & 0 & 0\\
1 & \lambda & 0\\
\hline
0 & 0 & 1
\end{array}\right).
\]
Com que només hi ha una recta invariant (fixa), aquesta només pot ser la recta que passa per $F$, $A$ i $B$.
\begin{enumerate}
	\item[(a)]	El lloc geomètric de les imatges d'un punt $p$ arbitrari per qualsevol $f_{\lambda}$ és el conjunt $\{z\equiv(z_1,z_2)\in A\ \vert\ \exists\lambda\in\mathbb{R}:z_1=\lambda p_1\wedge z_2=p_1+\lambda p_2\}$, pràcticament per definició. Fixem-nos en que la primera coordenada depèn de l'afinitat agafada i dels punts donats, doncs el primer vector del sistema de referència depèn de $f$ ja que s'agafa respecte $\vec{FA}$ (ha de complir $(\tilde{f}-\lambda\idn)(v_1)=\vec{FA}$) i la constant $\lambda$ és igual a la raó simple entre els tres punts. En canvi, la segona coordenada depèn \textit{només} dels punts donats $F,A,B$, ja que el segon vector de la base agafada és $\vec{FA}$ i $\lambda=(F,A,B)$. Per tant, el lloc geomètric resultant és una recta.
	\item[(b)]	Per demostrar això, veurem que podem trobar la homotècia ($h_{\alpha}$) en concret per la qual es satisfà el producte per la homologia especial $H_{\beta}$. Aquesta homologia especial té per matriu de la part lineal, en la nostra referència,
	\[
	\begin{pmatrix}
	1&0\\
	\beta&1
	\end{pmatrix},
	\]
	i és la que busquem doncs per qualsevol punt de la recta $\left<F,A\right>$, la seva imatge és ell mateix, i per qualsevol altre punt, no; veiem-ho:
	\begin{itemize}
		\item $p\in \left<F,A\right>:$ Aleshores, $p=F+\mu_{p}\vec{FA}$, i $H_{\beta}(p)=H_{\beta}(F)+\tilde{H_{\beta}}(\mu_p\vec{FA})$. Calculant per la matriu,
		\[
		H_{\beta}(p)=F+\begin{pmatrix}
		1&0\\ \beta&1\end{pmatrix}\begin{pmatrix}
		0\\ \mu_p
		\end{pmatrix}=F+\begin{pmatrix}
		0\\ \mu_p
		\end{pmatrix}=F+\mu_p\vec{FA}=p.
		\]
		\item $p\not\in \left<F,A\right>:$ Aleshores, $p=F+\gamma_p v_1+\mu_p\vec{FA}$, i $H_{\beta}(p)=H_{\beta}(F)+\tilde{H_{\beta}}(\gamma_p v_1+\mu_p\vec{FA})$. Calculant per la matriu altre cop,
		\[
		H_{\beta}(p)=F+\begin{pmatrix}
		1&0\\ \beta&1\end{pmatrix}\begin{pmatrix}
		\gamma_p\\ 0
		\end{pmatrix}+\begin{pmatrix}
		1&0\\ \beta&1\end{pmatrix}\begin{pmatrix}
		0\\ \mu_p
		\end{pmatrix}		
		=F+\begin{pmatrix}
		\gamma_p\\ \beta\gamma_p
		\end{pmatrix}+\begin{pmatrix}
		0\\ \mu_p
		\end{pmatrix}\neq p.
		\]
	\end{itemize}
	
	Per tant, imposem la condició $f_{\lambda}=h_{\alpha}\circ H_{\beta}$. Representant totes aquestes afinitats en coordenades, tenim	
	\[
	\begin{pmatrix}
	\lambda&0\\ 1&\lambda
	\end{pmatrix}=\begin{pmatrix}
	\alpha&0\\
	0&\alpha
	\end{pmatrix}\begin{pmatrix}
	1&0\\
	\beta&1
	\end{pmatrix}=\begin{pmatrix}
	\alpha&0\\
	\alpha\beta&\alpha
	\end{pmatrix},
	\]
	i podem agafar $\alpha=\lambda,\beta=\dfrac{1}{\lambda}$ i tenim $f_{\lambda}$. $\alpha\neq0$ per definició d'homotècia, i $\beta$ el mateix per definició d'homologia.
\end{enumerate}
\end{document}
