\documentclass[11pt]{article}
\usepackage[utf8]{inputenc}
\usepackage{amsfonts}
\usepackage{amsmath,amssymb}
\usepackage{amsthm}
\usepackage{float}
\usepackage[margin=1.25in]{geometry}
\usepackage{color}
\usepackage{breqn}
\newcommand{\af}{\mathbb{A}}
\newcommand{\obs}{\underline{\textbf{Observació}}: }
\DeclareMathOperator{\nuc}{Nuc}
\DeclareMathOperator{\idn}{Id}
\author{Àlex Batlle Casellas}

\begin{document}
\begin{small}
Àlex Batlle Casellas
\end{small}\\
\paragraph{1.}	A la família d'afinitats de $\af^2_{\mathbb{R}}$ d'equacions
\[
\begin{cases}
	x'=ax+ay+b\\
	y'=ax+6y+b^2
\end{cases}
\]
hi ha quatre homologies els eixos de les quals són els costats d'un paral·lelogram. Determineu els vèrtexs d'aquest paral·lelogram.\\
\textbf{Resolució}\\
Podem expressar les afinitats corresponents com
\[
f_{a,b}(x,y)=\begin{pmatrix} a & a\\ a & 6\end{pmatrix}\begin{pmatrix} x\\ y\end{pmatrix}+\begin{pmatrix} b \\ b^2\end{pmatrix}=M\begin{pmatrix} x\\ y\end{pmatrix}+\hat{b}.
\]
Aleshores, els punts fixos compleixen $(M-\idn)\begin{pmatrix}x\\ y\end{pmatrix}+\hat{b}=\begin{pmatrix}0\\ 0\end{pmatrix}.$ Si escrivim això com a sistema lineal, trobarem valors d'$a$ i $b$ pels quals tenim rectes de punts fixos:
\[
\left(\begin{array}{cc|c}
a-1&a&-b\\
a&5&-b^2\\
\end{array}\right)\sim
\left(\begin{array}{cc|c}
-1&a-5&-b+b^2\\
0&a^2-5a+5&(a-1)b^2-ab\\
\end{array}\right)
\]
Si volem que aquest sistema tingui per solució una recta, la matriu ha de tenir rang 1, i per tant, la segona fila ha de ser tota de zeros. Per tant, resolem les equacions que ens surten d'igualar els elements de la segona fila a zero, és a dir,
\[
a^2-5a+5=0,\qquad (a-1)b^2-ab=0,
\]
de les que surten les solucions $a=\dfrac{5\pm\sqrt{5}}{2},b=0,\dfrac{5\pm\sqrt{5}}{3\pm\sqrt{5}}.$ En imposar aquests valors, podem agafar-ne exactament quatre combinacions, que fan una recta cada una.
\[
\begin{aligned}
r_1:\ -x+\dfrac{-5+\sqrt{5}}{2}y=0,\\
r_2:\ -x+\dfrac{-5-\sqrt{5}}{2}y=0,\\
r_3:\ -x+\dfrac{-5+\sqrt{5}}{2}y=\dfrac{5+\sqrt{5}}{3+\sqrt{5}},\\
r_4:\ -x+\dfrac{-5-\sqrt{5}}{2}y=\dfrac{5-\sqrt{5}}{3-\sqrt{5}}.
\end{aligned}
\]
Ara busquem les interseccions de les rectes que no són paral·leles i haurem trobat els vèrtexs del paral·lelogram.
\begin{itemize}
	\item $r_1\cap r_2:x=0,y=0.$
	\item $r_1\cap r_4:\begin{cases}
	-x+\dfrac{-5+\sqrt{5}}{2}y=0\\
	-x+\dfrac{-5-\sqrt{5}}{2}y=\dfrac{5-\sqrt{5}}{3-\sqrt{5}}
	\end{cases}\implies(\cdots)\implies x=-\sqrt{5},y=\dfrac{-1+\sqrt{5}}{3-\sqrt{5}}.$
	\item $r_3\cap r_2:\begin{cases}
	-x+\dfrac{-5-\sqrt{5}}{2}y=0\\
	-x+\dfrac{-5+\sqrt{5}}{2}y=\dfrac{5+\sqrt{5}}{3+\sqrt{5}}
	\end{cases}\implies(\cdots)\implies x=-\sqrt{5},y=\dfrac{1+\sqrt{5}}{3+\sqrt{5}}.$
	\item $r_3\cap r_4:\begin{cases}
	-x+\dfrac{-5+\sqrt{5}}{2}y=\dfrac{5+\sqrt{5}}{3+\sqrt{5}}\\
	-x+\dfrac{-5-\sqrt{5}}{2}y=\dfrac{5-\sqrt{5}}{3-\sqrt{5}}
	\end{cases}\implies(\cdots)\implies x=0,y=-1.$
\end{itemize}
Per tant, ja hem trobat els quatre vèrtexs, que són
\begin{itemize}
	\item[$A$]$=(0,0)$
	\item[$B$]$=(-\sqrt{5},\dfrac{-1+\sqrt{5}}{3-\sqrt{5}})$
	\item[$C$]$=(-\sqrt{5},\dfrac{1+\sqrt{5}}{3+\sqrt{5}})$
	\item[$D$]$=(0,1).$
\end{itemize}
\newpage

\paragraph{2.}	Siguin $F,A,B$ tres punts alineats del pla afí. Considerem totes les afinitats del pla que deixen fix $F$, transformen $A$ en $B$ i tenen una única recta fixa.
\begin{enumerate}
	\item[(a)] Trobeu el lloc geomètric de les imatges d'un punt donat per aquestes afinitats.
	\item[(b)] Demostreu que existeix una homotècia tal que les afinitats anteriors són el producte d'aquesta homotècia per les homologies especials d'eix $FA$.
\end{enumerate}
\textbf{Resolució}\\
\end{document}
