\documentclass[11pt]{book}
\usepackage[utf8]{inputenc}
\usepackage{amsfonts}
\usepackage{amsmath,amssymb}
\usepackage{amsthm}
\usepackage{float}
\usepackage[margin=1.25in]{geometry}
\usepackage{color}
\usepackage{breqn}
\newcommand{\kev}[1]{$\mathbb{#1}$-e.v.}
\newcommand{\propi}{\underline{\textbf{Propietats}}:\\}
\newcommand{\defi}{\underline{\textbf{Definició}}:\\}
\newcommand{\propo}{\underline{\textbf{Proposició}}:\\}
\newcommand{\dem}{\underline{\textbf{Demostració}}:\\}
\newcommand{\ej}{\underline{Exemples}:\\}
\newcommand{\obs}{\underline{\textbf{Observació}}: }
\newcommand{\cor}{\underline{\textbf{Corol·lari}}:\\}
\title{Àlgebra Lineal Numèrica}
\author{Àlex Batlle Casellas}
\renewcommand*\contentsname{Índex}
\DeclareMathOperator{\rg}{rg}
\DeclareMathOperator{\nuc}{Nuc}
\DeclareMathOperator{\idn}{Id}
\DeclareMathOperator{\diag}{Diag}
\newcommand{\field}[1]{\mathbb{#1}}
\newcommand{\norm}[1]{||#1||}
\newcommand{\matr}[2]{\mathcal{M}_{#1\times#2}(\field{R})}

\begin{document}

\setcounter{section}{-1}
\begin{titlepage}
	\centering
	{\scshape\LARGE Facultat de Matemàtiques i Estadística \par}
	\vspace{1cm}
	{\scshape\Large Universitat Politècnica de Catalunya - BarcelonaTech\par}
	\vspace{1.5cm}
	{\huge\bfseries Àlgebra Lineal Numèrica (Q2)
	\par}
	\vspace{2cm}
	{\Large\itshape Àlex Batlle Casellas\par}

	\vfill

% Bottom of the page
	{\large \today\par}
\end{titlepage}

%\section*{Resumen}

\vfill
\newpage\tableofcontents
\chapter{Aritmètica finita i control d'errors}
Representem els nombres en un sistema de numeració posicional dependent d'una certa base $b$. Per representar un nombre (amb un nombre finit de decimals), seguim el següent esquema:
\[
\left(d_pd_{p-1}\ldots d_0.d_{-1}\ldots d_{-q}\right)=d_pb^p+\cdots+d_1b^1+d_0+d_{-1}b^{-1}+\cdots+d_{-q}b^{-q}=\sum_{i=-q}^pd_ib^i.
\]
En un ordinador, representem els nombres en binari; anem a veure com representem els enters i els reals:
\paragraph{Enters}
Els enters els representem de la forma següent:\\
\begin{table}[h!]
    \centering
    \begin{tabular}{ |c|c|c|c|c| }
        \hline
        $d_{s-1}$ & $d_{s-2}$ & $\cdots$ & $d_1$ & $d_0$\\
        \hline
    \end{tabular}
\end{table}\\
Utilitzem $s$ bits, i el primer és pel signe ($d_{s-1}=1\rightarrow-,d_{s-1}=0\rightarrow+$). Per tant, els enters en un ordinador es representen com una suma així
\[
(-1)^{s-1}\sum_{i=0}^{s-2}d_i2^i.
\]
D'aquí, deduïm que el nombre màxim que podem representar és
\[
|N_{\textrm{màx}}|=\sum_{i=0}^{s-2}2^i=2^{s-1}-1.
\]
En C/C++, que és el llenguatge que utilitzarem, tenim els següent tipus de dades
\begin{table}[h!]
    \centering
    \begin{tabular}{ |c|c|c|c|c| }
        \hline
        \textbf{Type} & \textbf{Bytes} & \textbf{Bits} & $N_{\textrm{màx}}$\\
        \hline
        char & 1 & 8 & $2^7-1=127$\\
        \hline
        short & 2 & 16 & $2^{15}-1=32767$\\
        \hline
        int & 4 & 32 & $2^{31}-1=2147483647$\\
        \hline
        unsigned int & 4 & 32 & $2^{32}-1=4294967295$\\
        \hline
        float & 4 & 32 (M:24,E:8) & $1.7815\cdot10^{38}$\\
        \hline
        double & 8 & 64 (M:53,E:11) & $0.8988\cdot10^{308}$\\
        \hline
    \end{tabular}
\end{table}
\newpage
\chapter{Sistemes Lineals.}
\section{Descomposició QR.}
\subsection{Mètode d'ortogonalització modificat de Gram-Schmidt.}
Sigui $A\in\matr{n}{m},m\geq n$ de rang màxim $n$, volem calcular la descomposició $A=QR$ amb $Q\in\matr{m}{n}$ ortogonal ($Q^tQ=\idn$) i $R\in\matr{n}{n}$ triangular superior no singular.
\paragraph{Recordatori del mètode de Gram-Schmidt:} Donada una base $\{\vec{v_1},\ldots,\vec{v_n}\}$, volem una base ortonormal $\{\vec{u_1},\ldots,\vec{u_n}\}$:\\
$\vec{u_1}=\dfrac{\vec{v_1}}{\norm{\vec{v_1}}}\textrm{ unitari},$\\

\begin{enumerate}
    \item[\underline{Pas 0.}] Definim $A_1=A=(a_1^{(1)}a_2^{(1)}\cdots a_n^{(1)})$, on $a_j^{(1)}$ és la columna $j$-èssima (amb $m$ components d'$A_1$).
    \item Normalitzem la primera columna $a_1^{(1)}$:
    \[
    r_{11}=\norm{a_1^{(1)}}_2\quad q_1=\dfrac{a_1^{(1)}}{r_{11}}.
    \]
    Ara ortogonalitzem, respecte d'aquesta, totes les columnes posteriors:
    \[
    r_{1s}=q_1^ta_s^{(1)},\quad a_s^{(2)}=a_s^{(1)}-r_{1s}q_1,\quad s=2,\ldots,n.
    \]
    Obtenim la matriu $A_2=(q_1\ a_2^{(2)}\ldots a_n^{(2)})$. Es compleix $q_1^ta_s^{(2)}=q_1^ta_s^{(1)}-r_{1s}q_1^tq_1=r_{1s}-r_{1s}=0,\ s=2,\ldots,n,$ i el rang d'$A_2$ és $n$. En forma matricial,
    \[
    \begin{pmatrix}
        a_{11}^{(1)} & \ldots & a_{1n}^{(1)}\\
        \vdots & \ & \vdots\\
        \vdots & \ & a_{nn}^{(1)}
    \end{pmatrix}\diag\left(\dfrac{1}{r_{11}},1\ldots,1\right)=\begin{pmatrix}
    q_{11} & a_{12}^{(1)} & \ldots & a_{1n}^{(1)}\\
    \vdots & \vdots & \ & \vdots\\
    q_{m1} & a_{m2}^{(1)} & \ldots & a_{mn}^{(1)}
    \end{pmatrix};    
    \]
    Ara
    \[
    \begin{pmatrix}
    q_{11} & a_{21}^{(1)} & \vdots\\
    \vdots & \vdots & \vdots\\
    q_{m1} & \ldots & a_{mn}^{(1)}
    \end{pmatrix}\begin{pmatrix}
    1 & -r_{12} & -r_{13} & \ldots & -r_{1n}\\
    0 & 1 & 0 & \ldots & 0\\
    0 & 0 & 1 & \ldots & 0\\
    \vdots & \vdots & \vdots & \vdots & \vdots\\
    0 & \ldots & \ldots & \ldots & 1
    \end{pmatrix}=
    \]
    \item[\underline{Pas 2.}] 
    \item[\underline{Pas $k$.}] Suposem que tenim la matriu $A_k=(q_1q_2\ldots q_{k-1}a_k^{(k)}\ldots a_n^{(k)})$ que satisfà (per construcció)
    \[
    (\textrm{Q})\ q_j^tq_l=\delta_{jl},\ q_ja_s^{(k)}=0,\ j,l=1,\ldots,k-1,\ s=k,\ldots,n.
    \]
    Ara volem $A_{k+1}$:\\
    Normalitzem la columna $k$-èssima
    \[
    r_{kk}=\norm{a_k^{(k)}}_2,\quad q_k=\dfrac{a_k^{(k)}}{r_{kk}}
    \]
    i ortogonalitzem respecte d'aquesta les columnes $a_{k+1}^{(k)},\ldots,a_n^{(k)}$, i.e.,
    \[
    r_{ks}=q_k^ta_s^{(k)},\ a_s^{(k+1)}=a_s^{(k)}-r_{ks}q_k,\ s=k+1,\ldots,n.
    \]
    Així obtenim $A_{k+1}=(q_1q_2\cdots q_k a_{k+1}^{(k+1)}\cdots a_n^{(k+1)})$ i se satisfan les relacions (Q) amb $k+1$ enlloc de $k$:
    \[
    q_j^tq_l=\delta_{jl},q_j^ta_s^{(k+1)}=0,\ j,l=1,\ldots,k,\ s=k+1,\ldots, n.
    \]
    Es té $A_k=A_{k+1}R^{(k)}\implies A_1=A_kR^{(k-1)}\ldots R^{(1)}=A_{k+1}R^{(k)}R^{(k-1)}\ldots R^{(1)}.$
\end{enumerate}
Després d'$n$ passos obtenim
\[
A_{n+1}=(q_1q_2\ldots q_n),\ A_n=(q_1\ldots q_{n-1}a_n^{(n)}),
\]
que té el mateix rang $n$ i es compleix
\[
A=A_{n+1}R^{(n)}\cdots R^{(1)}=QR,
\]
on $Q\in\matr{n}{n}$ de columnes ortonormals, $R\in\matr{n}{m}$ triangular superior.
\newpage
\end{document}
