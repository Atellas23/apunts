\documentclass[11pt]{article}
\usepackage[utf8]{inputenc}
\usepackage{amsfonts}
\usepackage{amsmath,amssymb}
\usepackage{amsthm}
\usepackage{float}
\usepackage[margin=1.25in]{geometry}
\usepackage{color}
\usepackage{breqn}
\newcommand{\kev}[1]{$\mathbb{#1}$-e.v. }
\newcommand{\props}{\underline{\textbf{Propiedades}}:\\}
\newcommand{\defi}{\underline{\textbf{Definición}}:\\}
\newcommand{\prop}{\underline{\textbf{Proposición}}:\\}
\newcommand{\dem}{\underline{\textbf{Demostración}}:\\}
\newcommand{\ej}{\underline{Ejemplos}:\\}
\newcommand{\obs}{\underline{\textbf{Observación}}: }
\newcommand{\cor}{\underline{\textbf{Corolario}}:\\}
\title{Àlgebra Lineal Numèrica}
\author{Àlex Batlle Casellas}
\renewcommand*\contentsname{Índex}
\DeclareMathOperator{\rg}{rg}
\DeclareMathOperator{\nuc}{Nuc}
\newcommand{\field}[1]{\mathbb{#1}}

\begin{document}

\setcounter{section}{-1}
\begin{titlepage}
	\centering
	{\scshape\LARGE Facultat de Matemàtiques i Estadística \par}
	\vspace{1cm}
	{\scshape\Large Universitat Politècnica de Catalunya - BarcelonaTech\par}
	\vspace{1.5cm}
	{\huge\bfseries Àlgebra Lineal Numèrica (Q2)
	\par}
	\vspace{2cm}
	{\Large\itshape Àlex Batlle Casellas\par}

	\vfill

% Bottom of the page
	{\large \today\par}
\end{titlepage}

%\section*{Resumen}

\vfill
\newpage\tableofcontents\newpage
\section{Aritmètica finita i control d'errors}
Representem els nombres en un sistema de numeració posicional dependent d'una certa base $b$. Per representar un nombre (amb un nombre finit de decimals), seguim el següent esquema:
\[
\left(d_pd_{p-1}\ldots d_0.d_{-1}\ldots d_{-q}\right)=d_pb^p+\cdots+d_1b^1+d_0+d_{-1}b^{-1}+\cdots+d_{-q}b^{-q}=\sum_{i=-q}^pd_ib^i.
\]
En un ordinador, representem els nombres en binari; anem a veure com representem els enters i els reals:
\paragraph{Enters}
Els enters els representem de la forma següent:\\
\begin{table}[h!]
    \centering
    \begin{tabular}{ |c|c|c|c|c| }
        \hline
        $d_{s-1}$ & $d_{s-2}$ & $\cdots$ & $d_1$ & $d_0$\\
        \hline
    \end{tabular}
\end{table}\\
Utilitzem $s$ bits, i el primer és pel signe ($d_{s-1}=1\rightarrow-,d_{s-1}=0\rightarrow+$). Per tant, els enters en un ordinador es representen com una suma així
\[
(-1)^{s-1}\sum_{i=0}^{s-2}d_i2^i.
\]
D'aquí, deduïm que el nombre màxim que podem representar és
\[
|N_{\textrm{màx}}|=\sum_{i=0}^{s-2}2^i=2^{s-1}-1.
\]
En C/C++, que és el llenguatge que utilitzarem, tenim els següent tipus de dades
\begin{table}[h!]
    \centering
    \begin{tabular}{ |c|c|c|c|c| }
        \hline
        \textbf{Type} & \textbf{Bytes} & \textbf{Bits} & $N_{\textrm{màx}}$\\
        \hline
        char & 1 & 8 & $2^7-1=127$\\
        \hline
        short & 2 & 16 & $2^{15}-1=32767$\\
        \hline
        int & 4 & 32 & $2^{31}-1=2147483647$\\
        \hline
        unsigned int & 4 & 32 & $2^{32}-1=4294967295$\\
        \hline
        float & 4 & 32 (M:24,E:8) & $1.7815\cdot10^{38}$\\
        \hline
        double & 8 & 64 (M:53,E:11) & $0.8988\cdot10^{308}$\\
        \hline
    \end{tabular}
\end{table}
\newpage
\section{Sistemes Lineals.}
\newpage
\end{document}
