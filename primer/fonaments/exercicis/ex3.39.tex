\documentclass[11pt]{article}
\usepackage[utf8]{inputenc}
\usepackage{amsfonts}
\usepackage{amsmath}
\usepackage{amsthm}
\usepackage{amssymb}
\usepackage{float}
\usepackage[margin=3cm]{geometry}
\usepackage{color}
\usepackage{enumitem}
\title{Exercicis: Fonaments de les Matemàtiques}
\author{Àlex Batlle Casellas}
\renewcommand*\contentsname{Índex}
\newlist{legal}{enumerate}{10}
\setlist[legal]{label*=\arabic*.}
\newcommand{\pendent}{\textbf{\textcolor{red}{PENDENT D'ACABAR.}}\\}
\DeclareMathOperator{\ord}{ord}
\DeclareMathOperator{\mcm}{mcm}

\begin{document}

\begin{legal}
	\item[3.39.] \textbf{Demostra que si $\tau_1$ i $\tau_2$ són dues transposicions diferents, aleshores $\tau_1\tau_2$ és d'ordre 2 o 3.}\\
	Si són dues transposicions diferents, això vol dir que $\tau_1=(i,j)\neq\tau_2=(k,l)$. Aleshores,
	$$
	\tau_1\cap\tau_2=\begin{cases}
	\emptyset,\quad\textrm{si }\tau_1\textrm{ i }\tau_2\textrm{ no comparteixen cap element igual,}\\
	\{j\},\quad\textrm{ja que sense pèrdua de generalitat podem suposar que si }\\
	\qquad\quad\tau_1\textrm{ i }\tau_2\textrm{ comparteixen un element aquest pot ser }j.
	\end{cases}
	$$
	Si no comparteixen cap element, la composició de $\tau_1$ i $\tau_2$ és la següent:
	$$
	\tau_1\tau_2=(i,j)(k,l)=\begin{pmatrix}
	i & j & k & l\\
	j & i & l & k
	\end{pmatrix},
	$$
	que per definició té ordre igual al mínim comú múltiple dels ordres de ls transposicions composades (que són cicles disjunts). En aquest cas,
	$$
	\ord(\tau_1\tau_2)=\mcm(\ord{\tau_1},\ord{\tau_2})=\mcm(2,2)=2.
	$$
	En el cas que tinguin un element comú, suposem (sense pèrdua de generalitat) $j=k$ i en composar:
	$$
	\tau_1\tau_2=(i,j)(j,l)=\begin{pmatrix}
	i & j & l\\
	j & l & i
	\end{pmatrix}
	$$
	veiem que $\ord(\tau_1\tau_2)=3$.\\
	$\square$
\end{legal}
\end{document}
