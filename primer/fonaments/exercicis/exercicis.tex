\documentclass[11pt]{article}
\usepackage[utf8]{inputenc}
\usepackage{amsfonts}
\usepackage{amsmath}
\usepackage{amsthm}
\usepackage{amssymb}
\usepackage{float}
\usepackage[margin=1.25in]{geometry}
\usepackage{color}
\usepackage{enumitem}
\title{Exercicis: Fonaments de les Matemàtiques}
\author{Àlex Batlle Casellas}
\renewcommand*\contentsname{Índex}
\newlist{legal}{enumerate}{10}
\setlist[legal]{label*=\arabic*.}

\begin{document}

\begin{titlepage}
	\centering
	{\scshape\LARGE Facultat de Matemàtiques i Estadística \par}
	\vspace{1cm}
	{\scshape\Large Universitat Politècnica de Catalunya - BarcelonaTech\par}
	\vspace{1.5cm}
	{\huge\bfseries Exercicis resolts de Fonaments de les Matemàtiques (Primer curs del Grau de Matemàtiques)
	\par}
	\vspace{2cm}
	{\Large\itshape Àlex Batlle Casellas\par}

	\vfill

% Bottom of the page
	{\large \today\par}
\end{titlepage}

%\section*{Resumen}

\vfill
\newpage

\tableofcontents
\newpage
\section{}

\newpage

\section{Conjunts i aplicacions.}
\begin{legal}[start=21]
	\item Siguin $A_1,A_2,B_1,B_2\neq\emptyset$. Demostreu:
	\begin{legal}[start=3]
		\item $(A_1\cup A_2)\times (B_1\cup B_2)=(A_1\times B_1)\cup (A_1\times B_2)\cup (A_2\times B_1)\cup (A_2\times B_2)$:
		$$\textrm{Sigui }y\in (A_1\cup A_2)\times (B_1\cup B_2).\textrm{ Aleshores, }\exists y_1\in A_1\cup A_2, \ y_2\in B_1\cup B_2: \ y=(y_1,y_2).$$
		$$\iff (y_1\in A_1 \vee y_1\in A_2)\wedge (y_2\in B_1 \vee y_2\in B_2)\iff (y_1\in A_1\wedge y_2\in B_1)$$
		$$\vee (y_1\in A_2 \wedge y_2\in B_1)\wedge (y_1\in A_1\wedge y_2\in B_2)\vee (y_1\in A_2\wedge y_2\in B_2)$$
		$$\iff y\in (A_1\times B_1)\cup (A_2\times B_2)\cup (A_1\times B_2)\cup (A_2\times B_1).\square$$
	\end{legal}
	\item[30.] Considerem una aplicació $f: \ A\mapsto B$ i subconjunts $A',A''\subseteq A$ i $B',B''\subseteq B$. Demostreu:
	\begin{legal}
		\item[30.1.] Si $A'\subseteq A''$, aleshores $f(A')\subseteq f(A'')$. Demostreu que la igualtat és certa si $f$ és injectiva.
		$$\textrm{Sigui }A'\subseteq A''.\textrm{ Aleshores }f(A')=\{y\in B:(\exists x\in A':\ f(x)=y)\}$$
		$$\subseteq \{y\in B:(\exists x\in A'':f(x)=y)\}=f(A'')\implies f(A')\subseteq f(A'').\square$$
		Si $f$ és injectiva, volem veure que $f(A'')\subseteq f(A')$ (ja que la primera inclusió per la igualtat ja l'hem demostrada a l'apartat anterior).
		$$\textrm{Sigui }A'\subseteq A''.\textrm{ Aleshores }f(A'')\subseteq f(A') \iff A''\subseteq A'\textrm{ (resultat anterior) }\iff$$
		$$A''=A'\textrm{ (perquè sabem }A'\subseteq A''\textrm{) }\iff\textrm{ (sabent que }f(A')\subseteq f(A'')\textrm{) }f\textrm{ és injectiva.}\square$$
		\item[30.2.] Si $B'\subseteq B''$, aleshores $f^{-1}(B')\subseteq f^{-1}(B'')$. Demostreu que la igualtat és certa si $f$ és exhaustiva.
		$$\textrm{Sigui }B'\subseteq B''.\textrm{ Aleshores, }f^{-1}(B')=\{x\in A:(\exists y\in B':f^{-1}(\{y\})=\{x\})\}\subseteq$$
		$$\{x\in A:(\exists y\in B'':f^{-1}(\{y\})=\{x\})\}=f^{-1}(B'')\implies f^{-1}(B')\subseteq f^{-1}(B'').\square$$
		Si $f^{-1}$ és exhaustiva, volem veure que $f^{-1}(B'')\subseteq f^{-1}(B')$ (ja que la primera inclusió per la igualtat ja l'hem demostrada a l'apartat anterior) quan la igualtat dels dos conjunts $B'$ i $B''$ es dóna, és a dir, quan $B'\subseteq B''$ i $B''\subseteq B'$. Quan passa això, per l'anterior demostració:
		\begin{itemize}
			\item $f^{-1}(B')\subseteq f^{-1}(B''),$
			\item $f^{-1}(B'')\subseteq f^{-1}(B').$
		\end{itemize}
		Per tant, tenim que $f^{-1}(B')=f^{-1}(B'')$.
		
%		Això (l'exhaustivitat) vol dir que ambdós conjunts $B'$ i $B''$ tenen un conjunt antiimatge que els cobreix sencers. De fet, significa que (sabent $B'\subseteq B''$) el conjunt antiimatge $f^{-1}(B'')$.
		
		
		
%		La igualtat d'antiimatges es donarà quan $B'=B''$.
		%\subseteq f^{-1}(B') \iff B''\subseteq B'\textrm{ (resultat anterior) }\iff$$
		%$$B''=B'\textrm{ (perquè sabem }B'\subseteq B''\textrm{) }\iff\textrm{ (sabent }f^{-1}(B')\subseteq f^{-1}(B'')\textrm{) }f\textrm{ és exhaustiva.}\square$$
	\end{legal}
	\item[31.] Considerem una aplicació $f: \ A\mapsto B$. Demostreu:
	\begin{legal}
		\item[31.1.] Si $A'\subseteq A$, aleshores $A'\subseteq f^{-1}(f(A'))$.
		\item[31.2.] $f$ és injectiva si i només si $A'=f^{-1}(f(A')) \ \forall A'\subseteq A$.
	\end{legal}
\end{legal}

\newpage

\section{Relacions, operacions i estructures.}
\end{document}
