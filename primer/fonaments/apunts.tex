\documentclass[11pt]{article}
\usepackage[utf8]{inputenc}
\usepackage{amsfonts}
\usepackage{amsmath}
\usepackage{amsthm}
\usepackage{amssymb}
\usepackage{float}
\usepackage[margin=1.25in]{geometry}
\usepackage{color}
\newcommand{\propietats}{\underline{{\scshape Propietats:}}}
\newcommand{\definicio}{\underline{\textbf{Definició:}}\\}
\newcommand{\proposicio}{\underline{\textbf{Proposició:}}\\}
\newcommand{\demostracio}{\textbf{{\scshape Demostració:}}\\}
\newcommand{\ex}{\underline{Exemple:}}
\newcommand{\pendent}{\textcolor{red}{\textbf{PENDENT D'ACABAR.}}}

\title{Fonaments de les Matemàtiques}
\author{Àlex Batlle Casellas}
\renewcommand*\contentsname{Índex}

\begin{document}

\begin{titlepage}
	\centering
	{\scshape\LARGE Facultat de Matemàtiques i Estadística \par}
	\vspace{1cm}
	{\scshape\Large Universitat Politècnica de Catalunya - BarcelonaTech\par}
	\vspace{1.5cm}
	{\huge\bfseries Apunts de Fonaments de les Matemàtiques (Primer curs del Grau de Matemàtiques)
	\par}
	\vspace{2cm}
	{\Large\itshape Àlex Batlle Casellas\par}

	\vfill

% Bottom of the page
	{\large \today\par}
\end{titlepage}

%\section*{Resumen}

\vfill
\newpage

\tableofcontents
\newpage
\section{}
\section{Conjunts i aplicacions.}
\section{Relacions, operacions i estructures.}
\definicio $R$ és una \textit{relació binària} en un conjunt $A$ si $R\subseteq A\times A$.
\propietats \begin{itemize}
	\item \textbf{Reflexiva:} $\forall x\in A (xRx)$.
	\item \textbf{Simètrica:} $\forall x,y\in A (xRy \rightarrow yRx)$.
	\item \textbf{Antisimètrica:} $\forall x,y\in A (xRy\wedge yRx\rightarrow x=y)$.
	\item \textbf{Transitiva:} $\forall x,y,z\in A (xRy\wedge yRz\rightarrow xRz)$.
	\item \textbf{Connexa:} $\forall x,y\in A (xRy\vee yRx)$.
\end{itemize}
\subsection{Relacions d'equivalència.}
\definicio Una relació $R$ en un conjunt $A\neq\emptyset$ s'anomena \textit{d'equivalència} si compleix les propietats reflexiva, simètrica i transitiva.
\end{document}
