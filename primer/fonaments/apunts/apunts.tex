\documentclass[11pt]{article}
\usepackage[utf8]{inputenc}
\usepackage{amsfonts}
\usepackage{amsmath}
\usepackage{amsthm}
\usepackage{amssymb}
\usepackage{float}
\usepackage[margin=1.25in]{geometry}
\usepackage{color}
\newcommand{\propietats}{\underline{{\scshape Propietats:}}}
\newcommand{\obs}{\underline{Observació:} }
\newtheorem{defi}{Definició}[section]
\newtheorem{prop}{Proposició}[section]
\newtheorem{thm}{Teorema}[section]
\newtheorem{ax}{Axioma}[section]
\newcommand{\demostracio}{\textbf{{\scshape Demostració:}}\\}
\newcommand{\ex}{\underline{Exemple:}}
\newcommand{\pendent}{\textcolor{red}{\textbf{PENDENT D'ACABAR.}}}


\title{Fonaments de les Matemàtiques}
\author{Àlex Batlle Casellas}
\renewcommand*\contentsname{Índex}

\begin{document}

\begin{titlepage}
	\centering
	{\scshape\LARGE Facultat de Matemàtiques i Estadística \par}
	\vspace{1cm}
	{\scshape\Large Universitat Politècnica de Catalunya - BarcelonaTech\par}
	\vspace{1.5cm}
	{\huge\bfseries Apunts de Fonaments de les Matemàtiques (Primer curs del Grau de Matemàtiques)
	\par}
	\vspace{2cm}
	{\Large\itshape Àlex Batlle Casellas\par}

	\vfill

% Bottom of the page
	{\large \today\par}
\end{titlepage}

%\section*{Resumen}

\vfill
\newpage

\tableofcontents

\newpage

\setcounter{section}{1}

\section{Conjunts i aplicacions.}
\begin{ax}
Axioma d'extensionalitat. $$A=B\iff\forall x(x\in A \leftrightarrow x\in B).$$
\end{ax}
\begin{defi}
Relació d'inclusió.
$$B\subseteq A\iff \forall x(x\in B\rightarrow x\in A).$$
\end{defi}
\noindent\propietats
\begin{enumerate}
	\item $A\subseteq A;$
	\item $A\subseteq B\wedge B\subseteq C\implies A\subseteq C;$
	\item $A\subseteq B\wedge B\subseteq A \iff A=B;$
	\item $\forall A, \ \emptyset\subseteq A.$
\end{enumerate}
\underline{Inclusió estricta:}
\begin{enumerate}
	\item $A\not\subset A;$
	\item $B\subset A\implies A\not\subset B;$
	\item $A\subset B\wedge B\subset C\implies A\subset C;$
	\item $A\neq\emptyset\iff\emptyset\subset A.$
\end{enumerate}
\subsection{Operacions amb conjunts.}
\begin{defi}
Unió d'$A$ i $B$ ($A\cup B$).
$$A\cup B=\{x:x\in A\vee x\in B\}.$$
\end{defi}
\noindent\propietats
\begin{enumerate}
	\item $A\cup A=A;$
	\item $A\cup B=B\cup A;$
	\item $(A\cup B)\cup C=A\cup (B\cup C)=A\cup B\cup C;$
	\item $A\cup\emptyset=A;$
	\item $A\subseteq A\cup B, \ B\subseteq A\cup B;$
	\item $A\subseteq B\iff A\cup B=B;$
	\item $A\cup B\subset C\iff A\subseteq C\wedge B\subseteq C.$
\end{enumerate}
\begin{defi}
Intersecció d'$A$ i $B$ ($A\cap B$).
$$A\cap B=\{x:x\in A\wedge x\in B\}.$$
\end{defi}
\noindent\propietats
\begin{enumerate}
	\item $A\cap A=A;$
	\item $A\cap B=B\cap A;$
	\item $(A\cap B)\cap C=A\cap (B\cap C)=A\cap B\cap C;$
	\item $A\cap\emptyset=\emptyset;$
	\item $A\cap B\subseteq A, \ A\cap B\subseteq B;$
	\item $A\subseteq B\iff A\cap B=A;$
	\item $C\subset A\cap B\iff C\subseteq A\wedge C\subseteq B.$
\end{enumerate}
\begin{defi}
Diferència d'$A$ i $B$ ($A-B$ o $A\backslash B$).
$$A\backslash B=\{x:x\in A\wedge x\not\in B\}.$$
\end{defi}
\noindent\propietats
\begin{enumerate}
	\item $A-\emptyset=A,\emptyset-A=\emptyset,A-A=\emptyset;$
	\item $A-B\subseteq A;$
	\item $(A-B)\cap B=\emptyset;$
	\item $A\subseteq B\iff A-B=\emptyset;$
	\item $C\subseteq A-B\iff (C\subseteq A)\wedge(C\cap B=\emptyset).$
\end{enumerate}
\underline{{\scshape Propietats de la unió i la intersecció:}}
\begin{enumerate}
	\item $A\cap(B\cup C)=(A\cap B)\cup(A\cap C);$
	\item $A\cup(B\cap C)=(A\cup B)\cap(A\cup C);$
	\item $A\cap(A\cup B)=A;$
	\item $A\cup(A\cap B)=A.$
\end{enumerate}
\begin{defi}
Conjunt complementari. Fixem un conjunt $\Omega$ i considerem només subconjunts d'$\Omega$. El complementari d'un subconjunt $A$ d'$\Omega$ és el conjunt de tots els elements d'$\Omega$ que no pertanyen a $A$. (Notació: $A^c$ o $\bar{A}$):
$$A^c=\{x\in\Omega:x\not\in A\}=\Omega-A.$$
\end{defi}
\propietats
\begin{enumerate}
	\item $\emptyset^c=\Omega,\ \Omega^c=\emptyset;$
	\item $(A^c)^c=A;$
	\item $A\cap A^c=\emptyset;$\\
	$A\cup A^c=\Omega;$
	\item $A\subseteq B\iff B^c\subseteq A^c;$
	\item $A\cap B=\emptyset\iff A\subseteq B^c\iff B\subseteq A^c;$
	\item $A\cup B=\Omega\iff A^c\subseteq B\iff B^c\subseteq A;$
	\item $A-B=A\cap B^c;$
	\item $(A\cup B)^c=A^c\cap B^c.$
\end{enumerate}
\begin{defi}
Parell ordenat. El parell ordenat de $x$ i $y$ és un objecte que denotem per $(x,y)$ que compleix:
$$(x,y)=(z,t)\iff x=z \wedge y=t.$$
Definició de Kuratowski:
$$(x,y)=\{\{x\},\{x,y\}\}.$$
\end{defi}
\begin{defi}
Producte cartesià. El producte cartesià de dos conjunts $A,B$ és el conjunt format per tots els parells ordenats $(x,y)$ tals que $x\in A$ i $y\in B$. (Notació: $A\times B$)
$$A\times B=\{(x,y):x\in A\wedge y\in B\};$$
anàlogament,
$$A_1\times A_2\times\cdots\times A_n=\{(x_1,x_2,\ldots,x_n):x_i\in A_i\forall i\}.$$
\end{defi}
\propietats
\begin{enumerate}
	\item $A\times\emptyset=\emptyset=\emptyset\times A;$
	\item $A\times B=B\times A\iff A=B\vee A=\emptyset\vee B=\emptyset.$
\end{enumerate}
\subsection{Conjunt de les parts.}
\begin{defi}
Conjunt de les parts. Anomenem el conjunt de les parts d'$A$ el conjunt que té per elements tots els subconjunts d'$A$. Notació: $\mathcal{P}(A)$.
$$\mathcal{P}(A)=\{X:X\subseteq A\}$$
\end{defi}
\noindent\propietats
\begin{enumerate}
\item $X\in\mathcal{P}(A)\iff X\subseteq A;$
\item $\emptyset\in\mathcal{P}(A),\ A\in\mathcal{P}(A).$
\end{enumerate}
\subsection{Aplicacions.}
\begin{defi}
Correspondència. Una correspondència és una terna $(A,B,G)$ on $A$ i $B$ són conjunts i $G\subseteq A\times B$.
\end{defi}
\begin{defi}
Aplicació. Una aplicació és una correspondència $(A,B,f)$ on $f\subseteq A\times B$:
$$\forall x\in A \ \exists!y\in B:(x,y)\in f.$$
Anomenem a $f(x)=y$ la imatge d'$x$ per $f$.
\end{defi}
\noindent Notació:
$$f: \ A\mapsto B.$$
$$A\xrightarrow{f}B.$$
\noindent Al conjunt $A$ l'anomenem domini, i al conjunt $B$ codomini.\\
\begin{defi}
Restricció. Donada una aplicació $f: \ A\mapsto B$ i un subconjunt $A'\subseteq A$, anomenem la restricció d'$f$ per $A'$ a l'aplicació $f_{|A'}: \ A'\mapsto B$.
\end{defi}
\begin{defi}
Aplicació identitat. L'aplicació identitat en un conjunt $A$ està definida per
$$I_A:\ A\mapsto A \quad I_A(x)=x\ \forall x\in A.$$
\end{defi}
\begin{defi}
Conjunt imatge. Si $f: \ A\mapsto B$ i $A'\subseteq A$, aleshores
$$f(A')=\{y\in B:\exists a\in A'(y=f(a))\}=\{f(a):a\in A'\}$$
és el conjunt imatge d'$A'$ per $f$.
\end{defi}
\begin{defi}
Conjunt antiimatge. Si $f: \ A\mapsto B$ i $B'\subseteq B$, aleshores
$$f^{-1}(B')=\{x\in A:f(x)\in B'\}\subseteq A$$
és el conjunt antiimatge de $B'$ per $f$.
\end{defi}
\noindent\textbf{Una qüestió de notació:} notem per $f^{-1}$ tant \textit{el conjunt antiimatge} com \textit{la funció inversa}. És important saber distingir entre aquests dos significats:
\begin{itemize}
\item $f^{-1}(\{b\})$ és el \textit{conjunt antiimatge} del conjunt $\{b\}$ per $f$.
\item $f^{-1}(b)$ pot ser (fent un abús de notació) el conjunt antiimatge del conjunt que té per únic element a $b$, com s'indica a 1., però també pot ser l'aplicació inversa (definida més endavant), que no sempre existeix.
\end{itemize}
\subsubsection{Injectivitat, exhaustivitat i bijectivitat.}
\begin{defi}
Injectivitat. $f: \ A\mapsto B$ és injectiva si i només si
$$\forall a_1,a_2\in A(a_1\neq a_2\implies f(a_1)\neq f(a_2)).$$
Se sol utilitzar el recíproc,
$$\forall a_1,a_2\in A(f(a_1)=f(a_2)\implies a_1=a_2).$$
\end{defi}
\begin{defi}
Exhaustivitat. $f: \ A\mapsto B$ és exhaustiva si i només si
$$\forall b\in B\exists a\in A:f(a)=b.$$
\end{defi}
\noindent\obs
\begin{enumerate}
\item $|A|>|B|:$ no hi ha aplicacions injectives $A\mapsto B$. Si n'hi hagués, $|A|\leq|B|$.
\item $|B|>|A|:$ no hi ha aplicacions exhaustives $A\mapsto B$. Si n'hi hagués, $|A|\geq|B|$.
\end{enumerate}
\begin{defi}
Bijectivitat.  $f: \ A\mapsto B$ és exhaustiva si i només si $f$ és injectiva i exhaustiva.
\end{defi}
\noindent\obs $A,B$ finits i existeix una bijecció.
$$A\mapsto B\implies (|A|\leq|B|)\wedge (|A|\geq|B|)\implies |A|=|B|.$$
Aleshores, $f$ és bijectiva si i només si $|A|=|B|$. Donat un $y\in B$,
\begin{itemize}
\item $f$ és injectiva $\implies\exists x\in A:f(x)=y.$
\item $f$ és exhaustiva $\implies\exists!x\in A:f(x)=y.$
\end{itemize}
\begin{defi}
Aplicació inversa. L'aplicació inversa d'una bijecció $f$ és aquella aplicació que a cada membre del codomini li assigna l'antiimatge per $f$.
$$\forall y\in B\exists!x\in A:f(x)=y$$
Es nota $f^{-1}: \ B\mapsto A$. Si $f$ bijectiva, aleshores
$$f(x)=y\iff f^{-1}(y)=x$$
\end{defi}
\noindent\propietats
\begin{enumerate}
	\item L'aplicació inversa és única;
	\item $f$ bijectiva $\iff f^{-1}$ bijectiva $\wedge (f^{-1})^{-1}=f;$
	\item Si $f:\ A\mapsto B$ i $g:\ B\mapsto C$ són bijectives, aleshores $g\circ f$ és bijectiva i $(g\circ f)^{-1}=f^{-1}\circ g^{-1};$
	\item $f$ bijectiva, aleshores $f\circ f^{-1}=I_B,\ f^{-1}\circ f=I_A;$
	\item $f:\ A\mapsto B$ és bijectiva $\iff\exists!\ g:\ B\mapsto A$ tal que $g\circ f=I_A$ i $f\circ g=I_B$. En tal cas, $f$ i $g$ són inverses mútuament.
\end{enumerate}
\subsubsection{Composició d'aplicacions.}
\begin{defi}Composició d'aplicacions. Si $f:\ A\mapsto B$ i $g:\ B\mapsto C$ són aplicacions, la composició de $f$ i $g$ és l’aplicació $g\circ f:\ A \mapsto C$ tal que $(g\circ f)(a)=g(f(a))$, per a tot $a\in A$.$$A\xrightarrow{f}B\xrightarrow{g}C$$ $$a\rightarrow f(a)\rightarrow g(f(a))$$\end{defi}
\noindent\propietats
\begin{enumerate}
	\item Associativitat. Si $f:\ A\mapsto B$, $g:\ B\mapsto C$ i $f:\ C\mapsto D$, aleshores
	$$h\circ(g\circ f)=(h\circ g)\circ f;$$
	\item No commutativitat. En general, la composició d'aplicacions no és commutativa. Si $f:\ A\mapsto A$ i $g:\ A\mapsto A$, no sempre és cert que
	$$f\circ g=g\circ f;$$
	\item Si $f:\ A\mapsto B$, aleshores $I_B\circ f=f=f\circ I_A.$
\end{enumerate}
\underline{{\scshape Propietats de la composició d'aplicacions: relació amb la injectivitat i l'exhaustivitat.}}
\begin{enumerate}
	\item $f$ i $g$ injectives $\implies g\circ f$ injectiva;
	\item $g\circ f$ injectiva $\implies f$ injectiva;
	\item $g\circ f$ injectiva i $f$ exhaustiva $\implies g$ injectiva;
	\item $f$ i $g$ exhaustives $\implies g\circ f$ exhaustiva;
	\item $g\circ f$ exhaustiva $\implies g$ exhaustiva;
	\item $g\circ f$ exhaustiva i $g$ injectiva $\implies f$ exhaustiva;
	\item $f$ i $g$ bijectives $\implies g\circ f$ bijectiva;
	\item $g\circ f$ bijectiva $\implies g$ exhaustiva i $f$ injectiva.
\end{enumerate}
\begin{defi}
	
\end{defi}

\newpage

\section{Relacions, operacions i estructures.}
\begin{defi}
	$R$ és una \textit{relació binària} en un conjunt $A$ si $R\subseteq A\times A$. \\
\end{defi}
\noindent\propietats \begin{itemize}
	\item \textbf{Reflexiva:} $\forall x\in A (xRx)$.
	\item \textbf{Simètrica:} $\forall x,y\in A (xRy \rightarrow yRx)$.
	\item \textbf{Antisimètrica:} $\forall x,y\in A (xRy\wedge yRx\rightarrow x=y)$.
	\item \textbf{Transitiva:} $\forall x,y,z\in A (xRy\wedge yRz\rightarrow xRz)$.
	\item \textbf{Connexa:} $\forall x,y\in A (xRy\vee yRx)$.
\end{itemize}


\subsection{Relacions d'equivalència.}
\begin{defi} Una relació $R$ en un conjunt $A\neq\emptyset$ s'anomena \textit{d'equivalència} si compleix les propietats reflexiva, simètrica i transitiva.\end{defi}
\begin{defi} Definim la \textit{classe d'equivalència} d'un element $x\in A$ com:
$$[x]_R=\{y\in A|yRx\}.$$
També escrivim $[x]$ o $\bar{x}$ quan no hi ha risc de confusió.\end{defi}
\propietats \begin{enumerate}
	\item $\forall x\in A(x\in [x]).$
	\item $\forall x,y\in A (xRy\iff [x]=[y]).$
	\item $A=\bigcup_{x\in A}[x].$
\end{enumerate}
\begin{defi} Anomenem una \textit{partició d'un conjunt} a una família $\Pi$ de subconjunts d'$A$ i diferents del buit, disjunts dos a dos, tals que la seva unió és tot $A$. És a dir, $\Pi\subseteq\mathcal{P}(A)$.\end{defi}
\propietats
\begin{enumerate}
	\item $X\neq\emptyset$ $\forall X\in\Pi.$
	\item $X\cap Y = \emptyset$ si $X,Y\in\Pi,X\neq Y.$
	\item $A=\bigcup_{X\in\Pi}X.$
\end{enumerate}
Els subconjunts $X\in\Pi$ s'anomenen les \textit{parts} o \textit{blocs de la partició}.\\
\begin{defi} Anomenem el \textit{conjunt quocient} d'un altre conjunt $A$ respecte la relació $R$ al conjunt format per totes les classes d'equivalència definides a partir d'$R$.$$A/R=\{\alpha|\exists x\in A([x]=\alpha)\}.$$\end{defi}
\begin{prop} El conjunt quocient $A/R$ és una partició d'$A$.\end{prop}
\noindent\propietats
\begin{enumerate}
	\item Tota relació d'equivalència definida en un conjunt $A$ indueix una partició d'$A$: el conjunt quocient $A/R$;
	\item Recíprocament, associada a tota partició $\Pi$ d'$A$ definim una relació $R_\Pi$ en $A$:
	$$xR_\Pi y\iff\exists B\in\Pi:x\in B\wedge y\in B;$$
	\item La relació $R_\Pi$ és d'equivalència.
\end{enumerate}
\begin{prop} Relacions i particions.
\begin{enumerate}
	\item Si $R$ és una relació d'equivalència en $A$, llavors $R_{A/R}=R$;
	\item Si $\Pi$ és una partició d'$A$, llavors $A/R_\Pi=\Pi.$
\end{enumerate}
\end{prop}
\subsubsection{Descomposició canònica d'una aplicació.}
Sigui $f:\ A\mapsto B$ una aplicació. Definim a $A$ la relació:
$$xR_fy\iff f(x)=f(y).$$
\begin{prop}
$R_f$ és una relació d'equivalència a $A$.
\end{prop}
\noindent Definim les aplicacions:
\begin{equation}
\pi:\ A\mapsto A/R_f,\quad\pi(x)=[x]_{R_f}.
\end{equation}
\begin{equation}
\bar{f}:\ A/R_f\mapsto f(A),\quad\bar{f}([x]_{R_f})=f(x).
\end{equation}
\begin{equation}
i:\ f(A)\mapsto B,\quad i(y)=y.
\end{equation}
$$\implies f=i\circ\bar{f}\circ\pi.$$
\subsection{Relacions d'ordre.}
Sigui $\leq$ una relació binària en un conjunt $A$.
\begin{itemize}
	\item La relació $\leq$ és un \textit{preordre} si és reflexiva i transitiva. Es diu que $(A,\leq)$ és un \textit{conjunt preordenat}.
	\item La relació $\leq$ és un \textit{ordre parcial} si és un preordre amb la propietat antisimètrica. Es diu que $(A,\leq)$ és un \textit{conjunt parcialment ordenat}.
	\item La relació $\leq$ és un \textit{ordre total} si és un ordre parcial amb la propietat connexa. Es diu que $(A,\leq)$ és un \textit{conjunt totalment ordenat}.
\end{itemize}
\pendent Falten definicions de mínim, màxim, minimal i maximal.
\noindent
\begin{defi}
Un conjunt parcialment ordenat $(A,\leq)$ està \textbf{ben ordenat} si tot subconjunt $X\subseteq A,X\neq\emptyset$ té un mínim.
\end{defi}

\end{document}