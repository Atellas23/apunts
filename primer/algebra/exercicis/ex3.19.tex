\documentclass[11pt]{article}
\usepackage[utf8]{inputenc}
\usepackage{amsfonts}
\usepackage{amsmath}
\usepackage{amsthm}
\usepackage{amssymb}
\usepackage{float}
\usepackage[margin=3cm]{geometry}
\usepackage{color}
\usepackage{enumitem}
\title{Exercicis: Àlgebra Lineal}
\author{Àlex Batlle Casellas}
\renewcommand*\contentsname{Índex}
\newlist{legal}{enumerate}{10}
\setlist[legal]{label*=\arabic*.}
\newcommand{\pendent}{\textbf{\textcolor{red}{PENDENT D'ACABAR.}}\\}
\DeclareMathOperator{\nuc}{Nuc}
\DeclareMathOperator{\img}{Im}
\DeclareMathOperator{\End}{End}

\begin{document}

\begin{small}
Àlex Batlle Casellas
\end{small}\\

\begin{legal}
\item[3.19.] Sigui $f$ un endomorfisme d'un espai vectorial $E$.
	\item[(a) ]\textbf{Proveu que $\img{f^{n+1}}\subset\img{f^n}$ i $\nuc{f^n}\subset\nuc{f^{n+1}}$ per a tot nombre natural $n$.}\\
	Com que $f\in\End{f}$, $E$ és $f$-invariant, és a dir, que $f(E)\subseteq E$. També tenim que, per ser endomorfisme, $f(f(E))\subseteq f(E)$, o el que és el mateix, que $\img{f^2}\subseteq\img{f}$. Fem un procés inductiu sobre $n$ per veure que 
	\begin{equation}
		\img{f^{n+1}}\subset\img{f^n}
	\end{equation}
	és compleix:
	\begin{enumerate}
		\item \underline{Cas base:} $n=0$\\
		En aquest cas, veiem que $\img{f^{0+1}\subseteq\img{f^0}}=E$ Com ja hem vist, $f(E)\subseteq E$, i per tant és cert per $n=0$.
		\item \underline{Hipòtesi d'inducció:}
		\begin{equation}
			\forall n\leq n_0\in\mathbb{N}\ \img{f^{n+1}}\subseteq\img{f^n}.
		\end{equation}
		\item \underline{Pas inductiu:} volem veure que (2) és compleix per tota $n>n_0$. Per veure-ho, construïm l'aplicació lineal $f^{n+1}$:
		$$
		\begin{array}{lrr}
			f^{n+1}:\ \img{f^{n}}\longrightarrow E\\
			\qquad\qquad\qquad v\longmapsto f^{n+1}(v):=f(v)
		\end{array}.
		$$
		Com que $f$ és un endomorfisme, necessàriament $f(V)\subseteq V\ \forall V\subseteq E$ i $f^{n+1}(V)\subseteq\img{f^n}\ \forall n>n_0,V\subseteq\img{f^n}$. Per tant, $f^{n+1}(\img{f^n})\subseteq\img{f^n}\ \forall n\in\mathbb{N}$.$\square$\\
		%nucli n subconjunt de nucli n+1
		Pel cas del nucli, tenim:
$$		\begin{array}{lll}
		\nuc{f}\subseteq E \\
		\nuc{f^2}\subseteq\img{f}\\
		\ldots\\
		\nuc{f^n}\subseteq\img{f^{n-1}}
		\end{array}
$$
%de fet
		Utilitzant el resultat anterior:
		$$
		\img{f^{n+1}}\subseteq\img{f^n}\subseteq\img{f^{n-1}}
		$$
		Ja que $\img{f^{k-1}}$ és l'espai de sortida de l'aplicació $f^k$ per tota $k$ natural. Ara volem veure:
		\begin{equation}
		\nuc{f^n}\subseteq\nuc{f^{n+1}}\ \forall n\in\mathbb{N}.
		\end{equation}
		Farem també un procés inductiu sobre $n$ per veure aquest resultat:
		\begin{enumerate}
		\item \underline{Cas base:} $n=0$
		\end{enumerate}
	\end{enumerate}
	\item[(b) ]\textbf{Demostreu que, si $E$ té dimensió finita, existeix un natural $m$ tal que $\img{f^n}=\img{f^m}$ i $\nuc{f^n}=\nuc{f^m}$ per a tot $n\geq m$.}\\
	\item[(c) ]\textbf{Proveu, donant un contraexemple a l’espai de polinomis $\mathbb{R}[x]$, que l’apartat (b) no és cert si $E$ no té dimensió finita.}\\
\end{legal}
\end{document}