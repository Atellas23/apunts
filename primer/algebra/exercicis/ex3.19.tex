\documentclass[11pt]{article}
\usepackage[utf8]{inputenc}
\usepackage{amsfonts}
\usepackage{amsmath}
\usepackage{amsthm}
\usepackage{amssymb}
\usepackage{float}
\usepackage[margin=3cm]{geometry}
\usepackage{color}
\usepackage{enumitem}
\title{Exercicis: Àlgebra Lineal}
\author{Àlex Batlle Casellas}
\renewcommand*\contentsname{Índex}
\newlist{legal}{enumerate}{10}
\setlist[legal]{label*=\arabic*.}
\newcommand{\pendent}{\textbf{\textcolor{red}{PENDENT D'ACABAR.}}\\}
\DeclareMathOperator{\nuc}{Nuc}
\DeclareMathOperator{\img}{Im}

\begin{document}

\begin{small}
Àlex Batlle Casellas
\end{small}\\

\begin{legal}
\item[3.19.] Sigui $f$ un endomorfisme d'un espai vectorial $E$.
	\item[(a) ]\textbf{Proveu que $\img{f^{n+1}}\subset\img{f^n}$ i $\nuc{f^n}\subset\nuc{f^{n+1}}$ per a tot nombre natural $n$.}\\
	\item[(b) ]\textbf{Demostreu que, si $E$ té dimensió finita, existeix un natural $m$ tal que $\img{f^n}=\img{f^m}$ i $\nuc{f^n}=\nuc{f^m}$ per a tot $n\geq m$.}\\
	\item[(c) ]\textbf{Proveu, donant un contraexemple a l’espai de polinomis $\mathbb{R}[x]$, que l’apartat (b) no és cert si $E$ no té dimensió finita.}\\
\end{legal}
\end{document}